\documentclass[11pt,a4paper]{article}

% ---- Márgenes
\usepackage[a4paper, left=2cm, right=2cm, top=2.5cm, bottom=2.5cm]{geometry}

% ---- Idioma y fuentes (pdfLaTeX)
\usepackage[utf8]{inputenc}
\usepackage[T1]{fontenc}
\usepackage[spanish, es-tabla]{babel}

% ---- “Arial-like” con pdfLaTeX: Helvetica
\usepackage[scaled=0.92]{helvet}
\renewcommand{\familydefault}{\sfdefault}

% ---- Matemáticas y notación cuántica
\usepackage{amsmath,amsfonts,amssymb}
\usepackage{braket} % <-- Añadido: \ket, \bra, \braket
\usepackage{pdfpages}

% ---- Tablas
\usepackage{tabularx,booktabs}
\usepackage{array,longtable}
\renewcommand{\arraystretch}{1.4}
\newcolumntype{L}[1]{>{\raggedright\arraybackslash}p{#1}}
\newcolumntype{C}[1]{>{\centering\arraybackslash}p{#1}}
\newcolumntype{B}[1]{>{\raggedright\arraybackslash}p{#1}}

% ---- Gráficos
\usepackage{graphicx}
\usepackage{subcaption}
\usepackage{float}
\usepackage[table,xcdraw]{xcolor}
\definecolor{verdePastel}{RGB}{204, 235, 197}
\definecolor{verdeFresco}{RGB}{60,160,60}

% ---- Estilo y utilidades
\usepackage{fancyhdr}
\setlength{\headheight}{14pt}
\usepackage{csquotes}
\usepackage{parskip}

% ---- URLs más “rompibles”
\usepackage{xurl}

% ---- Interlineado 1.5
\usepackage{setspace}

% ---- Bibliografía (biber)
\usepackage[
  backend=biber,
  style=numeric,
  sorting=none,
  maxbibnames=99
]{biblatex}
\addbibresource{referencias.bib}

% ---- Hipervínculos (cargar al final)
\usepackage{hyperref}
\hypersetup{
  pdfauthor={Dustin Randel De Lara García},
  pdftitle={PEC2 Grupo A - Importancia del experimento de la doble rendija en la evolución de la Física},
  pdfkeywords={Física, doble rendija, mecánica cuántica, UNED},
  colorlinks=true,
  linkcolor=verdeFresco,
  urlcolor=verdeFresco,
  citecolor=verdeFresco
}

% ---- “Tabla” en lugar de “Cuadro”
\addto\captionsspanish{\renewcommand{\tablename}{Tabla}}

% ---- Encabezados/pies
\pagestyle{fancy}
\rhead{UNED}
\lhead{PEC2 Grupo A - Doble rendija}
\cfoot{\thepage}

% ---- Para reducir overfull en bibliografía (URLs + notas largas)
\setlength{\emergencystretch}{2em}

\begin{document}

% ---- Portada
\begin{center}
  {\includegraphics[width=0.5\textwidth]{icono_uned.jpg}\par}
  \vspace{1cm}
  {\bfseries\LARGE Universidad Nacional de Educación a Distancia \par}
  \vspace{1cm}
  {\scshape\Large Facultad de Educación\par}
  \vspace{0.5cm}
  {\scshape\Large Máster en Formación del Profesorado\par}
  \vspace{0.5cm}
  {\scshape\Large Evolución Histórica de las Ideas de la Física y de la Química \par}
  \vspace{1.5cm}
  {\scshape\Huge Importancia del experimento de la doble rendija en la evolución de la Física \par}
  \vspace{1.5cm}
  {\itshape\Large PEC 2 \par}
  \vfill
  {\Large Dustin Randel De Lara García \href{mailto:ddelara3@alumno.uned.es}{ddelara3@alumno.uned.es}\par}
  \vfill
  {\Large \today \par}
  \thispagestyle{empty}
\end{center}

\newpage

% ---- Interlineado 1.5 (a partir de aquí)
\onehalfspacing

\tableofcontents

\newpage

\section*{Resumen previo y enfoque}
\addcontentsline{toc}{section}{Resumen previo y enfoque}

Este ensayo defiende una tesis histórica concreta: el experimento de la doble rendija se ha convertido, más que en una “paradoja onda-partícula”, en un \emph{dispositivo de reorganización conceptual}. Cada vez que la Física ha intentado fijar qué entiende por realidad, qué puede exigir como predicción y qué significa medición, el esquema de la doble rendija ha funcionado como banco de pruebas: obliga a abandonar una lectura puramente clásica de trayectoria, a aceptar la predicción estadística basada en amplitudes y a reconocer que medir no es “mirar” pasivamente, sino intervenir sobre el sistema.

El objetivo no es defender una interpretación única, sino justificar por qué este experimento ha reordenado, de forma repetida, esas tres nociones y cómo tales reordenaciones impactaron en la Física de cada época y siguen sosteniendo principios actuales. Para que el hilo sea cronológico y coherente con el desarrollo del trabajo, el ensayo avanza de lo óptico a lo cuántico:

\begin{itemize}
\item \textbf{Contextualización histórica (siglo XIX):} cómo Young plantea la interferencia como \emph{criterio experimental} en el debate sobre la naturaleza de la luz, qué condiciones físicas exige (coherencia, fase y superposición) y qué consecuencias inmediatas tuvo para la óptica.
\item \textbf{De la luz a la materia (siglo XX):} por qué repetir el mismo esquema con electrones cambia las reglas: la propuesta de de Broglie y la aparición de franjas en $P_{12}(x)$ fuerzan a replantear qué significa “descripción ondulatoria” cuando los impactos son discretos.
\item \textbf{Respuesta cuántica:} cómo la mecánica cuántica reformula la predicción mediante la función de onda y la regla de Born, y por qué la medición de cuál-camino elimina la interferencia al seleccionar una alternativa (colapso/proyección en la lectura estándar).
\end{itemize}

\section{Contexto histórico y motivación del experimento de Young}

A comienzos del siglo XIX, la óptica no estaba cerrada conceptualmente: junto a la óptica geométrica de
rayos (muy eficaz para describir reflexión y refracción) persistía la discusión sobre qué descripción
física de la luz era adecuada, y empezaban a ganar peso enfoques que requerían tomar en serio
magnitudes que la geometría de rayos no necesita, en particular la \emph{fase} y la \emph{superposición}.
En ese contexto, Young no presenta la interferencia como una mera ``curiosidad'', sino como un
\emph{criterio experimental}: diseñar situaciones en las que las franjas no puedan entenderse como una
simple suma de aportes independientes, sino como el resultado necesario de la superposición de dos
\emph{porciones de luz} capaces de \emph{reforzarse} o \emph{anularse} \cite{Young1804}.

El objetivo explícito del trabajo de Young (1804) es precisamente ofrecer una \emph{demostración experimental}
de esa ``ley general de la interferencia'': que las franjas se producen por la interferencia de dos porciones
de luz derivadas de un mismo origen. La ambición es doble: (i) fijar un hecho robusto
``la existencia de mínimos estables por cancelación'' y (ii) usarlo como punto de apoyo para discutir
qué rasgos debe incorporar una teoría física de la luz; en particular, Young invoca y reinterpreta las
franjas descritas por Grimaldi y señala críticamente que tales fenómenos habían sido pasados por alto
en la tradición newtoniana \cite{Young1804}.


\subsection{Del Sol ``incoherente'' a dos porciones correlacionadas: coherencia espacial}
El obstáculo práctico es que la interferencia exige un desfase relativo estable: no basta con tener dos
orificios; hay que garantizar \emph{coherencia espacial}. Por desfase relativo se entiende la diferencia de fase $\Delta\phi$ entre las dos contribuciones que llegan a un mismo punto de la pantalla. Si una superficie con dos orificios se ilumina directamente con luz solar, cada orificio recibe la suma de contribuciones de múltiples puntos del disco solar con fases no correlacionadas, por lo que el patrón de interferencia se lava al promediar en el tiempo. En formulación didáctica estándar, la solución de Young consiste en anteponer una primera
pantalla $S_1$ con un \emph{único} orificio que actúa como fuente cuasi puntual, iluminando después los
dos orificios/rendijas de $S_2$; así, ambas aberturas de $S_2$ quedan excitadas por \emph{la misma}
porción del frente de onda y se consigue la \emph{coherencia espacial} generando en la pantalla $F$ un patrón de interferencia (Fig.~\ref{fig:doble_rendija_young}) \cite{Oliver_T13_Interferencia}.

\begin{figure}[H]
  \centering
  \includegraphics[width=0.4\textwidth]{doblerendija.png}
  \caption{Esquema moderno del montaje de doble rendija con fuente puntual previa
  \cite{Oliver_T13_Interferencia}.}
  \label{fig:doble_rendija_young}
\end{figure}

\subsection{Qué hizo realmente Young: Montaje experimental y secuencia operativa}

Lo notable es que Young implementa el principio de \emph{coherencia espacial} con medios extremadamente simples, en una versión anterior a la esquema moderna (Fig.~\ref{fig:doble_rendija_young}). Su \emph{Experimento 1} (1804) puede descomponerse en una secuencia operativa clara \cite{Young1804}:

\begin{itemize}
  \item \textbf{Fuente primaria y dirección del haz:} parte de luz solar y la dirige a lo largo de una habitación (Young menciona el uso de un espejo para encauzar el haz).
  \item \textbf{Filtrado espacial (creación de una fuente efectiva casi puntual):} dispone un pequeño orificio en una contraventana y lo cubre con papel grueso perforado con una aguja fina. El objetivo físico es reducir el tamaño angular efectivo de la fuente que ilumina el montaje: la abertura diminuta actúa como \emph{fuente secundaria} y mejora la coherencia espacial.
  \item \textbf{Elemento de bipartición del frente de onda:} introduce en el haz una \emph{tira estrecha de cartón} (obstáculo) situada de modo que sus dos bordes queden iluminados.
  \item \textbf{Generación de dos contribuciones correlacionadas:} por \emph{difracción} en cada borde, parte del campo se desvía hacia la región de sombra. Cada borde funciona como un ``camino'' efectivo (o fuente secundaria) y ambos aportes alcanzan una región común de la pared/pantalla situada detrás.
  \item \textbf{Observación del patrón:} en la zona donde \emph{se solapan} ambas contribuciones aparece un sistema de franjas claras y oscuras (Fig.~\ref{fig:patroin_franjas_young}). La existencia de mínimos estables señala cancelación sistemática, no un mero descenso de intensidad.
  \begin{figure}[H]
  \centering
  \includegraphics[width=0.6\textwidth]{Patron_interferencia.png}
  \caption{Patrón de franjas de interferencia observado en el experimento de Young con luz blanca
  \cite{Oliver_T13_Interferencia}.}
  \label{fig:patroin_franjas_young}
  \end{figure}
  \item \textbf{Prueba de causalidad (bloqueo de un camino):} Young coloca una pequeña pantalla adicional para bloquear la luz difractada procedente de \emph{uno} de los dos bordes. Al hacerlo, las franjas en la región de sombra \emph{desaparecen}, aun cuando la contribución del otro borde sigue llegando.
\end{itemize}

La inferencia de Young es directa: las franjas no son una propiedad de cada contribución por separado, sino un efecto \emph{conjunto} del solapamiento de dos porciones de luz derivadas de un mismo origen y, por tanto, correlacionadas \cite{Young1804}. Dicho de otra manera, el montaje realiza una \emph{división del frente de onda}: se obtienen dos caminos coherentes a partir de una misma fuente efectiva. Esa es la condición física que convierte una fuente extensa como el Sol en dos aportes capaces de producir cancelaciones estables en la pantalla.

\subsection{Traducción moderna: por qué el montaje produce franjas}
El texto de Young (1804) no emplea el formalismo actual de campos complejos ni la definición moderna de
intensidad como promedio temporal de $|E|^2$. Sin embargo, es útil introducir una \emph{traducción moderna}
porque hace explícito qué ingrediente físico convierte el montaje de la doble rendija en un \emph{criterio}: la aparición
de un término cruzado que solo existe si se suman \emph{amplitudes con fase} \cite{Oliver_T13_Interferencia}.

En la descripción contemporánea, si al punto $x$ de la pantalla llegan dos contribuciones del campo óptico,
el campo total se escribe como:
\begin{equation}
E(x)=E_1(x)+E_2(x)
\label{eq:young_Esum}
\end{equation}
La magnitud registrada por el detector es la intensidad media, proporcional a $\langle |E(x)|^2\rangle_t$,
lo que conduce a
\begin{equation}
I(x)=I_1(x)+I_2(x)+2\sqrt{I_1(x)\,I_2(x)}\cos\!\big(\Delta\phi(x)\big),
\label{eq:young_Iinterf}
\end{equation}
donde $\Delta\phi(x)$ es el desfase relativo. El término $2\sqrt{I_1I_2}\cos(\Delta\phi)$ contiene la clave física del fenómeno: no es reducible a una suma
de intensidades independientes, sino que refleja que la magnitud que se combina es el campo (amplitud) con su
fase relativa; solo después se obtiene la intensidad observada \cite{Oliver_T13_Interferencia}.


\subsection{Conclusión histórica inmediata}
El resultado decisivo no fue “la luz es una onda” como eslogan, sino esto:
\begin{quote}
\emph{Para explicar franjas oscuras estables en una región donde llegan dos contribuciones de la misma luz, hay que aceptar que lo observable (intensidad) depende de una magnitud no observable directamente (fase) y de una ley de superposición de amplitudes.}
\end{quote}
Eso reorganiza qué cuenta como explicación en óptica: ya no basta la óptica geométrica (rayos), sino que hace
falta un objeto de cálculo que incorpore fase y superposición, capaz de conectar la geometría del montaje y la
longitud de onda con una distribución espacial de intensidades sobre la pantalla. La “doble rendija” queda así
instalada como banco de pruebas conceptual que reaparecerá, con materia en lugar de luz, en el siglo XX.


\section{De la luz a la materia: repetir la doble rendija con electrones}

Una vez que el experimento de Young fija la interferencia como criterio para la naturaleza ondulatoria
de la luz, el siguiente paso es preguntar si ese comportamiento es exclusivo de lo electromagnético
o si puede aparecer también en materia. La cuestión es conceptualmente fuerte: si un haz de electrones
produce máximos y mínimos análogos a los de la interferencia óptica, entonces la interferencia deja de
ser “propiedad de la luz” y pasa a ser un rasgo general de la descripción de sistemas microscópicos.

El puente histórico lo formula Louis de Broglie al proponer que a una partícula con momento lineal $p$
puede asociársele una longitud de onda $\lambda$:
\begin{equation}
\lambda=\frac{h}{p}.
\label{eq:debroglie}
\end{equation}
En esta hipótesis, la posibilidad de difracción e interferencia ya no depende de “ser luz” o “ser materia”,
sino de preparar condiciones en las que la escala geométrica del dispositivo y la coherencia del haz sean
compatibles con la longitud de onda asociada. Aunque la verificación histórica más temprana
se obtuvo mediante difracción en cristales (1927: Davisson--Germer y G.~P.~Thomson) \cite{UNAM_Clase7_Cuantica}, en este
trabajo adoptaremos deliberadamente el montaje de doble rendija con electrones como banco de pruebas: la aparición de franjas
en $P_{12}(x)$ ofrece una justificación especialmente transparente de la relación \emph{onda-corpúsculo}.

\subsection{Montaje con electrones y analogía de ``cajas'': tres situaciones y el origen del conflicto}

\paragraph{Montaje físico (electrones).}
Siguiendo la descripción de Feynman, el experimento se plantea con: (i) un cañón de electrones que emite
partículas hacia una barrera, (ii) una pantalla con dos aberturas (dos rendijas/agujeros) y (iii) un detector
detrás de la barrera (una pantalla o \emph{backstop}) que registra dónde llega cada electrón a lo largo de una
coordenada transversal $x$. El patrón observado no es un único impacto, sino la distribución
de conteos $P(x)$ construida al acumular muchos eventos. Con solo una abertura abierta se obtienen $P_1(x)$ y
$P_2(x)$; con ambas abiertas se obtiene $P_{12}(x)$ \cite{FeynmanIII_01}.

\paragraph{Nota histórica mínima.}
El montaje aparece en Feynman como experimento-prototipo en sus lecciones de Caltech de comienzos de los años sesenta,
pero su realización experimental “tipo Young” con electrones se fue consolidando justamente en esa misma época; en la literatura suele
citarse a Jönsson (1961) como el primer reporte de interferencia electrónica con rendijas múltiples fabricadas
artificialmente \cite{Rosa2008_MMP}.


Para facilitar el hilo posterior, vamos a describir el experimento como si el electrón tuviera que salir de una de dos \emph{cajas}: \textbf{Caja 1} (rendija 1) o \textbf{Caja 2} (rendija 2). La pantalla registra únicamente \emph{dónde} llega el electrón, no \emph{de qué caja} salió, salvo que añadamos un dispositivo de medida de cuál-camino. Con esta analogía podemos presentar de forma ordenada tres situaciones históricamente decisivas: las dos primeras alimentan la intuición de dualidad onda-corpúsculo; la tercera la rompe.

\subsubsection*{Situación 1: solo una caja abierta (comportamiento ``corpuscular'' compatible)}
Si abrimos solo una rendija (Caja 1 o Caja 2), los electrones llegan como impactos localizados y la estadística
resultante define $P_1(x)$ o $P_2(x)$. Hasta aquí nada obliga a abandonar una intuición clásica: cada electrón
produce un punto en la pantalla y el patrón se interpreta como dispersión/difracción asociada a esa abertura.

\begin{figure}[H]
  \centering
  \includegraphics[width=0.6\textwidth]{doble_rendija.001.jpeg}
  \caption{Diagrama análogo del montaje de doble rendija con electrones \cite{PerezVerde2020_DobleRendija}. Abriendo una rendija a la vez se obtienen las distribuciones $P_1(x)$ y $P_2(x)$.}
  \label{fig:doble_rendija_electrones_1}
\end{figure}

\subsubsection*{Situación 2: dos cajas abiertas sin preguntar ``cuál'' (aparecen franjas)}
Si abrimos ambas rendijas pero no instalamos ningún mecanismo que nos diga de qué caja salió cada electrón,
la distribución observada $P_{12}(x)$ muestra máximos y mínimos (franjas), es decir, \emph{interferencia}
(Fig.~\ref{fig:doble_rendija_electrones_2}). En términos de predicción, esto contradice la regla clásica de
``sumar contribuciones independientes'' (Ec.~\ref{eq:clasica}).

\begin{figure}[H]
  \centering
  \includegraphics[width=0.6\textwidth]{doble_rendija.003.jpeg}
  \caption{Diagrama análogo del montaje de doble rendija con electrones \cite{PerezVerde2020_DobleRendija}. Abriendo ambas rendijas se obtiene una distribución $P_{12}(x)$ con franjas. El símbolo rojo indica que no se registra información de cuál-camino (no sabemos de qué ``caja'' salió cada electrón).}
  \label{fig:doble_rendija_electrones_2}
\end{figure}

\begin{equation}
P_{12}(x)\approx P_1(x)+P_2(x).
\label{eq:clasica}
\end{equation}

Estas dos primeras situaciones alimentan la \emph{dualidad onda-corpúsculo}:
el electrón se detecta como impacto puntual (rasgo de ``partícula''), pero el conjunto de eventos presenta franjas (rasgo de ``onda'').

\subsubsection*{Situación 3: dos cajas abiertas \emph{y} medimos cuál (desaparecen las franjas)}
El punto crítico llega al intentar completar la intuición clásica con una tercera suposición:
``aunque haya interferencia, cada electrón \emph{realmente} pasó por una caja concreta; bastaría con mirar''. Cuando se introduce un dispositivo que registra por cuál rendija pasa el electrón, la interferencia desaparece y el patrón se vuelve compatible con una suma de contribuciones sin término de interferencia (Ec.~\ref{eq:clasica}).

\begin{figure}[H]
  \centering
  \includegraphics[width=0.6\textwidth]{doble_rendija.002.jpeg}
  \caption{Diagrama análogo del montaje de doble rendija con electrones \cite{PerezVerde2020_DobleRendija}. Con ambas rendijas abiertas pero registrando cuál-camino, la distribución $P_{12}(x)$ pierde las franjas. El símbolo verde indica que se conoce por qué ``caja'' (rendija) pasó cada electrón.}
  \label{fig:doble_rendija_electrones_3}
\end{figure}

El hecho de que la interferencia desaparezca al registrar cuál-camino no se deja reducir a
``el electrón es onda y partícula a la vez'' si esa frase se entiende como una simple suma de intuiciones clásicas. Las tres situaciones del montaje muestran que el problema no es elegir entre dos “modelos” (onda o corpúsculo), sino disponer de una teoría que indique qué objeto físico se combina para predecir y cómo interviene la medición.

Ese es precisamente el punto donde la Física se ve obligada a recurrir a la mecánica cuántica: la descripción fundamental ya no se formula con trayectorias o con intensidades que se suman, sino con un estado (función de onda) que puede estar en superposición y cuya relación con el experimento depende del tipo de medida que se implemente.


\section{La respuesta cuántica: función de onda y colapso}

En mecánica cuántica, el sistema se describe por un estado $|\psi\rangle$
(o, en representación de posición, por una función compleja $\psi(x,t)$).
La idea clave es que $\psi$ no se interpreta como una “onda material” clásica,
sino como una amplitud cuyo módulo al cuadrado fija la distribución de resultados
posibles al medir una magnitud (en particular, la posición: Ec.~\ref{eq:born_12}) \cite{Castellano2025_MC1}.
\begin{equation}
P(x)=|\psi(x)|^2.
\label{eq:born_12}
\end{equation}

La evolución del estado cuando no se realiza una medida viene dada por la ecuación
de Schrödinger (Ec.~\ref{eq:schrodinger_td_1d}), que permite propagar amplitudes desde
la región de las rendijas hasta la pantalla \cite{Castellano2025_MC1}:
\begin{equation}
i\hbar\,\frac{\partial}{\partial t}\,\psi(x,t)
=
\left[
-\frac{\hbar^2}{2m}\frac{\partial^2}{\partial x^2} + V(x,t)
\right]\psi(x,t).
\label{eq:schrodinger_td_1d}
\end{equation}
En lo que sigue escribiremos $\psi(x)$ para referirnos a la amplitud en la pantalla (en el instante de detección),
omitiendo $t$ para simplificar la notación.

Aplicado a la doble rendija: si ambas rendijas están abiertas y no se registra cuál-camino,
la mecánica cuántica describe el estado en la pantalla como una superposición de alternativas.
Si $\psi_1(x)$ y $\psi_2(x)$ representan la contribución asociada a la rendija 1 y a la rendija 2,
un estado físicamente posible es
\begin{equation}
\psi(x)=\alpha_1\,\psi_1(x)+\alpha_2\,\psi_2(x),
\label{eq:superposicion_general}
\end{equation}
donde $\alpha_1$ y $\alpha_2$ codifican los pesos relativos (en el caso ideal simétrico, suelen tomarse iguales).
La predicción observacional se obtiene con la regla de Born, de modo que la distribución en la pantalla viene dada por
\begin{equation}
P_{12}(x)=\frac{|\psi(x)|^2}{\int_{-\infty}^{\infty}|\psi(x')|^2\,dx'}
=
\frac{|\alpha_1\psi_1(x)+\alpha_2\psi_2(x)|^2}{\int_{-\infty}^{\infty}|\alpha_1\psi_1(x')+\alpha_2\psi_2(x')|^2\,dx'}.
\label{eq:normalizacion}
\end{equation}
Al desarrollar el numerador aparece explícitamente el término de interferencia:
\begin{equation}
|\psi(x)|^2
=|\alpha_1\psi_1(x)|^2+|\alpha_2\psi_2(x)|^2
+\alpha_1^\ast\alpha_2\,\psi_1^\ast(x)\psi_2(x)
+\alpha_1\alpha_2^\ast\,\psi_1(x)\psi_2^\ast(x).
\label{eq:interf_general}
\end{equation}
Los dos últimos términos (cruzados) contienen la interferencia: dependen de la fase relativa entre
$\psi_1$ y $\psi_2$ y pueden contribuir de forma constructiva o destructiva. En el caso simple
$\alpha_1=\alpha_2=1$, Ec.~\ref{eq:interf_general} se reduce a la forma compacta habitual,
\begin{equation}
|\psi_1(x)+\psi_2(x)|^2
=|\psi_1(x)|^2+|\psi_2(x)|^2+2\,\mathrm{Re}\!\left(\psi_1^\ast(x)\psi_2(x)\right).
\label{eq:interf}
\end{equation}

\cite{Castellano2025_MC1}

\subsection{Por qué medir cuál-camino elimina franjas: colapso y evolución posterior del estado}

La tercera situación (dos rendijas con información de cuál-camino) se interpreta, en la lectura
estándar, como una actualización del estado debida al acto de medida: el resultado selecciona
una alternativa y el estado se proyecta sobre ella (colapso/proyección).
En la analogía de las cajas:
\begin{itemize}
  \item Si la medición indica Caja 1, el estado efectivo pasa a ser $\psi(x)\to \psi_1(x)$ (normalizada).
  \item Si la medición indica Caja 2, el estado efectivo pasa a ser $\psi(x)\to \psi_2(x)$ (normalizada).
\end{itemize}
A partir de ahí, el estado continúa evolucionando y la predicción en la pantalla se calcula como si
la otra rendija estuviera cerrada, es decir,
\begin{equation}
P(x)=|\psi_1(x)|^2 \quad \text{o} \quad P(x)=|\psi_2(x)|^2.
\end{equation}
Así desaparece el término de interferencia porque ya no coexisten simultáneamente las dos alternativas
en el estado efectivo. En el lenguaje de Feynman: cuando el experimento es capaz de determinar “por dónde”
pasó, se pierde la interferencia y las probabilidades se suman \cite{FeynmanIII_01}.

\printbibliography[title={Anexo: bibliografía consultada (lista razonada)}]
\includepdf[pages=-]{DECLARACÓN_JURADA_DE_AUTORÍA.pdf}
\end{document}
