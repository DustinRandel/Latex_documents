\documentclass[12pt,a4paper]{article}

% ---- Márgenes
\usepackage[a4paper, left=2cm, right=2cm, top=2.5cm, bottom=2.5cm]{geometry}

% ---- Idioma y fuentes
\usepackage[utf8]{inputenc}
\usepackage[T1]{fontenc}
\usepackage[spanish, es-tabla]{babel}
\usepackage{lmodern}

% ---- Interlineado
\usepackage{setspace}   % <--- paquete para el interlineado
       % <--- interlineado 1.15 en todo el documento

% ---- Matemáticas y tablas
\usepackage{amsmath,amsfonts,amssymb}
\usepackage{tabularx,booktabs}
\usepackage{enumitem}

% ---- Índice con puntitos
\usepackage{tocloft}
\renewcommand{\cftsecleader}{\textcolor{verdeFresco}{\cftdotfill{\cftdotsep}}}


% ---- Gráficos
\usepackage{graphicx}
\usepackage{subcaption}
\usepackage{float}
\usepackage[table,xcdraw]{xcolor}
\definecolor{verdePastel}{RGB}{204, 235, 197}
\definecolor{verdeFresco}{RGB}{60,160,60}

% ---- Estilo y utilidades
\usepackage{fancyhdr}
\setlength{\headheight}{15pt}
\usepackage{csquotes}
\usepackage{parskip}   % separa párrafos sin sangría
\usepackage{array,longtable}
\renewcommand{\arraystretch}{1.4}
\newcolumntype{L}[1]{>{\raggedright\arraybackslash}p{#1}}
\newcolumntype{C}[1]{>{\centering\arraybackslash}p{#1}}
\newcolumntype{B}[1]{>{\raggedright\arraybackslash}p{#1}}
\newcolumntype{Y}{>{\raggedright\arraybackslash}X}

\usepackage{ltablex}
\keepXColumns                % <- importante para conservar las columnas X
\usepackage[table]{xcolor}
\definecolor{verdePastel}{HTML}{D6EAD2}
\newcolumntype{M}[1]{>{\raggedright\arraybackslash}p{#1}}

% ---- Bibliografía (mejor con biber)
\usepackage[
  backend=biber,
  style=apa,
  sorting=nyt
]{biblatex}
\DeclareLanguageMapping{spanish}{spanish-apa}
\addbibresource{referencias.bib}

% ---- Hipervínculos (cargar al final)
\usepackage{hyperref}
\hypersetup{
  pdfauthor={Dustin Randel De Lara García},
  pdftitle={TUTORÍA COMO PROMOTORA DE LA
CONVIVENCIA ENEDUCACIÓN SECUNDARIA},
  pdfkeywords={Tutoría, Educación secundaria, Convivencia, UNED, Illes Balears},
  colorlinks=true,
  linkcolor=verdeFresco,
  urlcolor=verdeFresco,
  citecolor=verdeFresco
}

% ---- “Tabla” en lugar de “Cuadro” (por si acaso)
\addto\captionsspanish{\renewcommand{\tablename}{Tabla}}

% ---- Encabezados/pies
\pagestyle{fancy}
\rhead{UNED}
\lhead{PEC:PCE - Tutoría como promotora de la convivencia en educación secundaria} 
\cfoot{\thepage}

\date{\today}

\begin{document}

% ---- Portada
\begin{center}
  {\includegraphics[width=0.5\textwidth]{icono_uned.jpg}\par}
  \vspace{1cm}
  {\bfseries\LARGE Universidad Nacional de Educación a Distancia \par}
  \vspace{1cm}
  {\scshape\Large Facultad de Educación\par}
  \vspace{0.5cm}
  {\scshape\Large Máster en Formación del Profesorado\par}
  \vspace{0.5cm}
  {\scshape\Large Procesos y Contextos Educativos \par}
  \vspace{1.5cm}
  {\scshape\Huge tutoría como promotora de la convivencia en educación secundaria \par}
  \vspace{1.5cm}
  {\Large Dustin Randel De Lara García \par}
  \vfill
  {\Large Email: ddelara3@alumno.uned.es \par}
  \vfill
  {\Large \today \par}
  \thispagestyle{empty}
\end{center}

% ---- Índice
\newpage
\tableofcontents
\newpage
\begin{spacing}{1.15}
\section{Introducción}\label{sec:intro}

La acción tutorial ocupa un lugar central en la educación secundaria, tanto por su dimensión académica como por su dimensión personal y social. Más allá de los trámites administrativos o del seguimiento de calificaciones, la tutoría se ha consolidado como un espacio específico para acompañar al alumnado en su trayectoria escolar, orientar sus decisiones y cuidar el clima de relaciones en el aula. Desde esta perspectiva, la función tutorial forma parte del trabajo docente y se vincula directamente con la calidad, la equidad y la inclusión del sistema educativo.

Al mismo tiempo, la convivencia escolar se ha convertido en una preocupación constante de los centros de secundaria. Los conflictos cotidianos, las dinámicas de exclusión o el impacto de las redes sociales en las relaciones entre iguales muestran que no basta con gestionar puntualmente los problemas más visibles. Es necesario generar espacios donde el alumnado pueda tomar conciencia de cómo se relacionan, aprender a comunicarse de manera más respetuosa y participar en la construcción de normas compartidas. La acción tutorial ofrece un escenario ideal para este trabajo, al situar en primer plano la dimensión relacional y emocional de la vida escolar.

La presente PEC se plantea en un cruce entre teoría, normativa y práctica. Su objetivo general es profundizar en el concepto y la práctica de la tutoría en educación secundaria, prestando especial atención a su contribución a la mejora de la convivencia en el aula y en el centro. De manera más específica, busca comprender el sentido educativo de la acción tutorial, analizar cómo se articula con otros elementos de la vida escolar y explorar qué implica asumir el rol de tutor o tutora en términos de acompañamiento, prevención y construcción de una escuela más inclusiva.

Estos objetivos se concretan en el desarrollo del trabajo en tres partes. En la primera, se revisa la fundamentación teórica y el marco normativo que sustentan la acción tutorial, así como el rol específico del tutor o tutora de grupo y el engranaje que proporciona el Plan de Acción Tutorial dentro del proyecto educativo del centro. La segunda se centra en el diseño y análisis de una propuesta concreta de intervención tutorial ante un reto de convivencia en un grupo de 3.º de ESO, vinculada a la comunicación interpersonal y digital, planteando un plan de actuación de carácter preventivo, inclusivo y participativo. Finalmente, en la tercera parte se recogen las principales conclusiones y aprendizajes derivados de la experiencia, reflexionando sobre lo que este proceso aporta a la propia identidad profesional docente.

\section{Primera parte: Fundamentación teórica y conceptual de la tutoría en
educación secundaria}\label{sec:parte-1}

\subsection{El concepto de tutoría y su sentido educativo en el ámbito de la convivencia}

Diversos autores coinciden en que la tutoría no es un añadido, sino un componente estructural del centro educativo. A través de ella se articulan las relaciones en el aula, la coordinación con las familias y el seguimiento del alumnado. Desde esta perspectiva, enseñar no puede reducirse a transmitir contenidos; docencia y tutoría forman un bloque conjunto, porque no tiene sentido enseñar física, historia o informática sin tener en cuenta a la persona que aprende y el contexto en el que lo hace \parencite{GarciaNieto1996,AlvarezGonzalez2017,LopezGomezQuijadaCap7}.

La tutoría se concibe como un proceso continuado de ayuda y acompañamiento que se adapta a la etapa, al grupo y a cada estudiante. Integra dimensiones académicas, personales y vocacionales, así como la relación con las familias, y se apoya en una planificación explícita con objetivos (Figura~\ref{fig:obj-tutoria}) y actuaciones definidas , más allá de intervenciones puntuales para “apagar fuegos” \parencite{GarciaNieto1996,AlvarezGonzalez2017,AlvarezJustel2017,LopezGomezQuijadaCap7}.

\begin{figure}[H]
    \centering
    \includegraphics[width=0.7\textwidth]{Modelo integrador de la tutoria.jpg}
    \caption{Objetivos del modelo integral de tutoría, \parencite{AlvarezGonzalez2017}.}
    \label{fig:obj-tutoria}
\end{figure}

Este enfoque integral permite comprender el papel de la tutoría en la convivencia. En línea con \textcite{MongeEtAlCap8}, la convivencia escolar se entiende como la construcción activa de relaciones basadas en el respeto, la participación y la justicia, y no como simple ausencia de conflictos. La hora de tutoría, las sesiones grupales y personales son espacios para trabajar habilidades socioemocionales, normas compartidas y formas dialogadas de gestionar los problemas cotidianos \parencite{LopezGomezQuijadaCap7}. En este marco, la mediación escolar se presenta como una estrategia concreta para abordar los conflictos desde la cooperación y el diálogo (Figura~\ref{fig:fases-mediacion}). Las propuestas de \textcite{Pantoja2005} y \textcite{Maso2022}, así como programas de convivencia, destacan precisamente esta dimensión preventiva y educativa frente a enfoques centrados solo en la sanción.

\begin{figure}[H]
    \centering
    \includegraphics[width=\textwidth]{Fases del proceso de mediación.jpg}
    \caption{Fases del proceso de mediación escolar, \parencite{MongeEtAlCap8}.}
    \label{fig:fases-mediacion}
\end{figure}

La relación entre tutoría y convivencia se refuerza si se considera la dimensión comunicativa e inclusiva de la acción educativa. \textcite{GonzalezLorenzoCap6} recuerda que toda relación pedagógica es un acto comunicativo: palabras, silencios, gestos y la organización del espacio transmiten el tipo de vínculo que el profesorado establece con el alumnado. \textcite{CamposGonzalezCap9} plantean, además, la atención a la diversidad desde un enfoque inclusivo que reconoce la pluralidad de capacidades, ritmos, culturas, géneros y lenguas presentes en cualquier grupo. Desde esta perspectiva, la acción tutorial se puede ver como entorno para visibilizar la diversidad, prevenir dinámicas excluyentes y construir climas en los que las diferencias se entienden como una riqueza y no como un problema.

Resumiendo, la tutoría es un proceso integral y compartido por el conjunto del centro, con un fuerte potencial para promover relaciones más justas, democráticas e inclusivas y para que el alumnado se sienta escuchado, reconocido y corresponsable del bienestar colectivo. En este contexto cobra sentido analizar con más detalle quién asume esta tarea en el día a día y cómo se concreta en la figura del tutor o tutora de grupo.

\subsection{El rol y la función del tutor o tutora}

En el sistema educativo español, el tutor o la tutora es un profesor del claustro que, además de impartir su materia, asume la responsabilidad específica de acompañar a un grupo de alumnos, coordinar la acción educativa sobre ese grupo y mantener la relación ordinaria con sus familias \parencite{RD83_1996}.

El \textcite{RD83_1996} establece que ``la tutoría y orientación de los alumnos forma parte de la función docente'' y que cada grupo debe contar con un profesor tutor, al que se le encomiendan tareas como participar en el Plan de Acción Tutorial, coordinar la evaluación del grupo y colaborar con el departamento de orientación. Desde una perspectiva pedagógica, \textcite{AlvarezGonzalez2017} considera la acción tutorial un componente básico del proceso educativo y puede entenderse como un proceso continuo y sistemático orientado a facilitar el aprendizaje, la toma de decisiones y la atención a la diversidad a lo largo de las distintas etapas.

A partir de las aportaciones de \textcite{Asensi2002}, \textcite{LopezGomezQuijadaCap7} y \textcite{VidalEtAl2023}, las funciones del tutor o tutora pueden agruparse en varios ámbitos que, en conjunto, definen su tarea:

\begin{itemize}
    \item \textbf{Con el alumnado a nivel individual}: conocer sus condiciones personales y familiares, detectar dificultades de aprendizaje, orientar problemas personales y escolares, acompañar las trayectorias y emitir el consejo orientador al final de etapa.
    
    \item \textbf{Con el grupo-clase}: seguir el rendimiento global, favorecer la integración y la cohesión, recoger iniciativas y sugerencias del alumnado, proporcionar métodos de trabajo y técnicas de estudio, y velar por el clima de aula y la convivencia cotidiana.
    
    \item \textbf{Con las familias}: conocer el contexto familiar del alumnado, informar y orientar a madres y padres sobre el proceso educativo, organizar entrevistas individuales y reuniones conjuntas y mantener canales de comunicación y colaboración estables.
    
    \item \textbf{Con el profesorado y los servicios de orientación}: intercambiar información relevante sobre el alumnado, coordinar las sesiones de evaluación y las actividades de recuperación, y mantener una comunicación fluida con el departamento de orientación y el resto del equipo docente.
    
    \item \textbf{Con el centro educativo en su conjunto}: conocer el contexto y el funcionamiento interno del centro, mantener una comunicación fluida con el equipo directivo, canalizar hacia él los problemas, necesidades y propuestas del grupo, y colaborar en el desarrollo y revisión del Plan de Acción Tutorial y de otros planes de mejora en coherencia con el proyecto educativo del centro.

\end{itemize}

Estas funciones requieren una coordinación estrecha entre tutores, equipo docente y servicios de orientación, de modo que la acción tutorial no quede reducida a actuaciones aisladas, sino integrada en la organización general del centro \parencite{LopezGomezQuijadaCap7}.

Además de estas funciones, diferentes trabajos señalan que la acción tutorial exige cierto perfil competencial. \textcite{AlvarezJustel2017} subraya que el profesor tutor debe disponer de diversas habilidades y estrategias profesionales y personales, entre las que destacan:

\begin{itemize}
    \item Ejercer un liderazgo cercano que acompañe al alumnado en su crecimiento
    \item Ser un buen comunicador y trabajar en red con otros agentes educativos y con las familias
    \item Observar y analizar para detectar necesidades, dificultades y potencialidades del alumnado
    \item Mostrar disposición a utilizar nuevas herramientas y recursos en la acción tutorial
    \item Desarrollar competencias emocionales (empatía, asertividad, capacidad de mediación y resiliencia)
    \item Implicarse, escuchar y respetar al alumnado en el trato cotidiano
    \item Atender a la diversidad, fomentando la cohesión del grupo y un clima relacional positivo en el aula y en el centro
    \item Asumir la formación permanente como parte del propio rol tutorial
\end{itemize}

El manual de \textcite{VidalEtAl2023} converge en la idea de que este conjunto de competencias resulta imprescindible para que la tutoría contribuya de manera efectiva a la educación integral, la calidad educativa y la inclusión.

En conjunto, estas aportaciones dibujan al tutor o tutora como una figura clave que orienta y acompaña al alumnado, coordina a los diferentes agentes educativos, actúa como puente con las familias y contribuye a que la acción tutorial se consolide como un eje imprescindible del proyecto de centro.

\subsection{Necesidad, relevancia y beneficios de la tutoría en ESO, Bachillerato y Formación Profesional}

Precisamente porque el tutor o la tutora asume este conjunto de funciones y competencias, conviene preguntarse por qué la acción tutorial es tan relevante en las distintas etapas de la educación secundaria. 

Retomando la definición de \textcite{AlvarezGonzalez2017}, la acción tutorial puede entenderse como un proceso continuo y sistemático de ayuda al alumnado en su aprendizaje y en la toma de decisiones. En la educación secundaria, esta función adquiere una relevancia particular, porque se convierte en una pieza clave para avanzar hacia una educación más integral, personalizada e inclusiva \parencite{AlvarezGonzalez2017, VidalEtAl2023}.

A partir de estas aportaciones, la necesidad y los beneficios de la tutoría pueden resumirse en cuatro grandes dimensiones:
\begin{itemize}
    \item \textbf{Para el alumnado}: ofrece un espacio específico para hablar de cómo aprende, de sus dificultades y de su proyecto de futuro. \textcite{AlvarezGonzalez2017} destaca que la tutoría ayuda a personalizar la educación, y los resultados de \textcite{LeonFernandez2019} apuntan a que el alumnado valora especialmente las tutorías que atienden tanto a lo académico como a lo personal.

    \item \textbf{Para el profesorado tutor}: permite coordinar mejor la información sobre el grupo y sobre cada estudiante, dar sentido a las decisiones de evaluación y ajustar la propia práctica docente \parencite{AlvarezJustel2017,LopezGomezQuijadaCap7}.
    
    \item \textbf{Para el centro}: cuando se organiza de forma colegiada, ayuda a detectar necesidades recurrentes, mejorar la convivencia y reforzar la atención a la diversidad \parencite{AlvarezJustel2017,VidalEtAl2023}.
    
    \item \textbf{Para las familias}: constituye el canal habitual de comunicación con el centro y facilita recibir orientación sobre el proceso educativo de sus hijos e hijas, aunque los estudios señalan que esta dimensión aún puede mejorar \parencite{LeonFernandez2019,Gonzalez2022}.
\end{itemize}

En la ESO, la importancia de la tutoría se hace especialmente visible en dos ámbitos. Por un lado, en la transición de Primaria a Secundaria, donde el cambio de centro, profesorado y exigencias puede generar inseguridad. \textcite{AvilaSanchezBueno2022} y \textcite{Bereziartua2017} señalan que los planes de transición apoyados en una buena acción tutorial facilitan la adaptación y reducen el riesgo de desafección. Por otro, en la convivencia diaria, distintos autores muestran que trabajar en tutoría habilidades sociales, resolución de conflictos y mediación contribuye a mejorar el clima de aula y a prevenir situaciones de violencia o acoso \parencite{MongeEtAlCap8,Pantoja2005,Maso2022,RojoFerrando2022,MEFPD2025Convivencia}.

En Bachillerato, la acción tutorial se vuelve especialmente relevante para la orientación académica y profesional. Las decisiones sobre itinerarios y estudios posteriores generan dudas e incertidumbre, y la tutoría grupal e individual ayuda a contrastar información y a elaborar un proyecto personal más realista \parencite{AlvarezJustel2017,MunozPastor2015}. En Formación Profesional, la tutoría cumple además una función de puente con el mundo laboral: el profesorado tutor coordina la formación en entornos de trabajo, mantiene la relación con las empresas colaboradoras y acompaña los procesos de inserción profesional del alumnado \parencite{GonzalezAlvarez2018FP,RD659_2023}.

Esta relevancia también aparece recogida en la normativa. La LOE y su
modificación por la LOMLOE, así como los reales decretos que regulan la ESO,
el Bachillerato y la nueva Formación Profesional, subrayan la orientación
educativa y profesional y la acción tutorial como elementos básicos del
sistema educativo, vinculándolos con los objetivos de calidad,
equidad e inclusión \parencite{LOE2006,LOMLOE2020,RD217_2022,RD243_2022,RD659_2023}.

\subsection{El marco normativo de la tutoría en educación secundaria, su engranaje en el PAT y su relación con la mejora de la convivencia, en el marco de los planes y programas institucionales}

La acción tutorial en educación secundaria no es solo una opción organizativa,
sino una exigencia recogida en la normativa básica del sistema. Las leyes
orgánicas de educación reconocen que la orientación educativa y profesional,
junto con la acción tutorial, forman parte de la función docente y constituyen
un instrumento para hacer efectivo el derecho del alumnado a una educación de
calidad, personalizada y equitativa, atenta a la diversidad y comprometida con
la mejora de la convivencia \parencite{LOE2006,LOMLOE2020}. En la
Tabla~\ref{tab:marco-normativo-tutoria} se recogen algunas de las normas
estatales que sostienen este marco.


\begin{longtable}{|B{3.5cm}|L{11.5cm}|}
\hline
\rowcolor{verdePastel}
\textbf{Normativa} & \textbf{Aportaciones sobre tutoría, orientación y convivencia} \\
\hline
\endfirsthead

\hline
\multicolumn{2}{|c|}{\textit{(Continuación de la página anterior)}}\\
\hline
\rowcolor{verdePastel}
\textbf{Normativa} & \textbf{Aportaciones sobre tutoría, orientación y convivencia} \\
\hline
\endhead

\hline
\multicolumn{2}{|r|}{\textit{(Continúa en la siguiente página)}}\\
\endfoot

\endlastfoot

LOE 2/2006 y LOMLOE 3/2020 &
Reconocen la orientación educativa y profesional y la acción tutorial como parte de la función docente, y las vinculan al derecho del alumnado a una educación de calidad. Relacionan el proyecto educativo, la atención a la diversidad y los planes de convivencia con una educación de calidad, equitativa e inclusiva \parencite{LOE2006,LOMLOE2020}. \\ \hline

Real Decreto 83/1996, de 26 de enero &
Establece que ``la tutoría y orientación de los alumnos forma parte de la función docente'', fija la existencia de un tutor por grupo y concreta funciones como la coordinación de la evaluación, la relación con las familias y la participación en el Plan de Acción Tutorial \parencite{RD83_1996}. \\ \hline

Real Decreto 217/2022, de 29 de marzo (ESO) &
Incorpora la tutoría personal y la orientación como elementos de la organización de la etapa. Introduce el consejo orientador al final de la ESO y relaciona la orientación con la prevención del abandono y la atención a la diversidad \parencite{RD217_2022}. \\ \hline

Real Decreto 243/2022, de 5 de abril (Bachillerato) &
Subraya la importancia de la tutoría y de la orientación académica y profesional para acompañar las decisiones del alumnado sobre itinerarios, modalidades y estudios posteriores, en coherencia con los principios de equidad e inclusión \parencite{RD243_2022}. \\ \hline

Real Decreto 659/2023, de 18 de julio (Sistema de FP) &
Insiste en que la orientación y la acción tutorial deben acompañar el proceso
formativo de cada persona. Regula la tutoría, el seguimiento personal y la
orientación profesional como parte integrada de las ofertas de FP y refuerza
el acompañamiento en las decisiones sobre el itinerario formativo y
profesional, con especial atención al alumnado con mayores dificultades de
aprendizaje o de inserción social y laboral \parencite{RD659_2023}. \\ \hline

\textit{Convivencia en acción} (MEFPD) &
Ofrece un marco de referencia para los planes de convivencia de los centros, con un enfoque preventivo y participativo. Señala el papel de la acción tutorial en la mejora del clima escolar, la gestión pacífica de los conflictos y la construcción de una cultura de convivencia democrática \parencite{MEFPD2025Convivencia}. \\ \hline

\caption{Marco normativo básico de la tutoría y la orientación en la educación secundaria y la Formación Profesional}
\label{tab:marco-normativo-tutoria}
\end{longtable}

A nivel de centro, \textcite{LopezGomezQuijadaCap7} sitúan el Plan de Acción Tutorial (PAT) como la pieza clave para articular este marco. El PAT se define como el documento que organiza y planifica el conjunto de actuaciones de tutoría y orientación: concreta objetivos, contenidos, actividades, responsables y procedimientos de evaluación, y los integra en el Proyecto Educativo del centro. De este modo, da coherencia a las funciones del profesorado tutor y evita que la acción tutorial quede reducida a iniciativas aisladas.

El PAT se relaciona estrechamente con otros planes y programas institucionales. Por un lado, con el Plan de Convivencia: \textcite{MongeEtAlCap8} subrayan que la convivencia escolar debe abordarse desde un enfoque preventivo y educativo, basado en la participación del alumnado, la construcción de normas compartidas y el uso de estrategias de gestión de conflictos como la mediación. En esta línea, propuestas como \textit{Convivencia en acción} reconocen la hora de tutoría y las actuaciones del profesorado tutor como espacios ideales para trabajar habilidades socioemocionales, resolución de conflictos y mediación \parencite{MEFPD2025Convivencia}.

Por otro lado, el PAT se vincula con las medidas de atención a la diversidad y con el enfoque inclusivo del centro. \textcite{CamposGonzalezCap9} plantean un continuo de actuaciones universales para responder a la diversidad del alumnado, en el que la tutoría desempeña un papel central de acompañamiento y seguimiento. El manual de \textcite{VidalEtAl2023} coincide en que la acción tutorial, bien planificada, contribuye a que el centro pueda ofrecer una educación más integral e inclusiva, conectando la orientación académica y profesional con el bienestar y la participación del alumnado.

En conclusión, el marco normativo establece la obligación de garantizar la tutoría y la orientación en las distintas etapas de la educación secundaria, y el PAT funciona como el engranaje que traduce y conduce estas exigencias legales en el Proyecto Educativo del centro. A través del PAT se coordinan las intervenciones del profesorado tutor, se enlazan los objetivos de convivencia y de inclusión con el Plan de Convivencia y las medidas de atención a la diversidad, y se concreta, en última instancia, de qué manera la acción tutorial contribuye a mejorar el clima escolar y la experiencia educativa del alumnado.

\section{Segunda parte: Planificación del desarrollo de algún contenido de la
acción tutorial}\label{sec:parte-2}

\subsection{Reto fundamental}

El contexto de referencia es un instituto público de educación secundaria situado en un barrio urbano de tamaño medio, con alumnado diverso en origen cultural, situación socioeconómica y trayectorias escolares. El plan se centra en 3.º de ESO B, con 29 estudiantes de entre 14 y 15 años.

A lo largo del primer trimestre, el equipo docente y la tutora han detectado diversas señales de mal clima de convivencia asociadas a la forma de comunicarse en el grupo:

\begin{itemize}
  \item Comentarios sarcásticos, motes y burlas sobre el aspecto físico, la manera de hablar o el rendimiento escolar.
  \item Conversaciones paralelas constantes, interrupciones y tono de voz elevado que dificultan el trabajo en el aula.
  \item Uso de móviles y grupos de mensajería para compartir memes, capturas de pantalla o mensajes que ridiculizan a ciertas personas.
  \item Respuestas evasivas ante los conflictos (``es broma'', ``no te lo tomes así'') que minimizan el daño y generan malentendidos.
\end{itemize}

No se ha identificado una situación clara de acoso continuado, pero sí microagresiones frecuentes y dinámicas de exclusión (no contar con determinados compañeros para trabajos en grupo, ignorarlos en actividades o en el patio) que afectan al bienestar emocional de parte del alumnado. En entrevistas individuales, algunos estudiantes expresan sentirse ``fuera de lugar'' o ``cansados de aguantar bromas''. Estos comportamientos se prolongan en el entorno digital y llegan a las familias de forma parcial, generando malestar y quejas puntuales.

En coherencia con lo trabajado en la primera parte, se considera que la tutoría, entendida como profesión de ayuda, es un espacio idóneo para abordar este reto. Desde la acción tutorial se puede:

\begin{itemize}
  \item Ayudar al grupo a tomar conciencia del impacto de su forma de comunicarse.
  \item Dotar al alumnado de herramientas para expresar desacuerdos sin dañar a los demás.
  \item Establecer, de manera participativa, un pacto de convivencia y de uso responsable de las TIC.
  \item Implicar a las familias y al resto del profesorado en un enfoque común.
\end{itemize}

El reto fundamental que se formula es el siguiente:

\begin{quote}
  \textbf{Mejorar el clima de convivencia de 3.º ESO B a través de la mejora de la comunicación interpersonal y digital, reduciendo las conductas de burla y exclusión y promoviendo interacciones basadas en el respeto, la escucha y la cooperación.}
\end{quote}

\subsection{Plan de actuación}

\subsubsection{Principios generales}

El plan de actuación se apoya en los siguientes principios generales:

\begin{enumerate}
  \item \textbf{Enfoque integral de la acción tutorial.}\\
  La intervención busca el desarrollo académico, personal, social y emocional del alumnado, ayudando al grupo a comunicarse mejor y a cada estudiante a conocerse, regularse y responsabilizarse de su forma de relacionarse.

  \item \textbf{Carácter preventivo y educativo.}\\
  Se interviene antes de que las dinámicas de burla y exclusión se consoliden, aprovechando conflictos reales para trabajarlos de forma educativa y no solo sancionadora.

  \item \textbf{Perspectiva inclusiva y de atención a la diversidad.}\\
  La diversidad lingüística, cultural, de género y de personalidad se entiende como una riqueza. Las actividades se diseñan para que pueda participar todo el alumnado, con distintas formas de expresión y niveles de habilidad social.

  \item \textbf{Comunicación como eje de convivencia.}\\
  Se parte de que toda relación educativa es comunicación; por ello se trabajarán explícitamente habilidades de escucha activa, mensajes en primera persona, empatía y gestión de conflictos.

  \item \textbf{Participación y corresponsabilidad.}\\
  El alumnado será protagonista del proceso (diagnóstico, propuestas, pacto de aula) y se implicará a familias y profesorado para favorecer la coherencia entre aula, centro y hogar.

  \item \textbf{Flexibilidad y evaluación continua.}\\
  Aunque se propone una secuenciación de cinco semanas, la tutora ajustará actividades, ritmos y apoyos según la respuesta del grupo y la información recogida.
\end{enumerate}

\subsubsection{Objetivos}

\paragraph{Objetivo general}

\begin{itemize}
  \item \textbf{OG.} Mejorar, en un periodo de cinco semanas, la calidad de la comunicación interpersonal y digital en 3.º ESO B, reduciendo la frecuencia de conductas de burla y exclusión y favoreciendo un clima de aula más respetuoso e inclusivo.
\end{itemize}

\paragraph{Objetivos específicos}

\begin{itemize}
  \item \textbf{OE1.} Identificar, de manera participativa, los principales problemas de comunicación y convivencia del grupo, tanto en el aula como en los espacios digitales.
  \item \textbf{OE2.} Tomar conciencia del impacto emocional de comentarios, gestos y mensajes en redes sobre el propio bienestar y el de los demás.
  \item \textbf{OE3.} Desarrollar habilidades básicas de comunicación asertiva y escucha activa, aplicándolas en simulaciones de conflicto y en situaciones reales.
  \item \textbf{OE4.} Elaborar y aprobar un \textbf{pacto de aula} sobre convivencia y uso responsable de las TIC, conocido por alumnado, familias y profesorado.
  \item \textbf{OE5.} Ofrecer espacios de \textbf{tutoría individual} al alumnado más implicado, ayudándole a reflexionar sobre su papel y a construir alternativas.
  \item \textbf{OE6.} Informar y sensibilizar a las familias sobre la importancia de la comunicación y el acompañamiento digital, acordando pautas conjuntas de seguimiento.
\end{itemize}

\subsubsection{Líneas de actuación, actividades y temporalización}

Se propone una intervención de cinco semanas articulada en torno a tres ejes:

\begin{itemize}
  \item Sesiones de tutoría grupal.
  \item Tutorías individuales con alumnado especialmente implicado o afectado por los conflictos.
  \item Tutorías con las familias y coordinación con el equipo docente.
\end{itemize}

A continuación se presenta la planificación agrupada por semanas, con una descripción sintética de cada sesión.

\paragraph*{Semana 1 (Sesiones 1 y 2). Diagnóstico del clima comunicativo.}

La primera semana se dedica a obtener una imagen inicial de la convivencia y de la forma de relacionarse en el grupo, combinando una sesión grupal y tutorías individuales con alumnado clave.

\begin{tabularx}{\textwidth}{|M{0.21\textwidth}|X|}
\hline
\rowcolor{verdePastel}
\multicolumn{2}{|c|}{\textbf{Sesión 1: Cómo nos hablamos en clase (tutoría grupal).}}\\
\hline
\textbf{Temporalización:} 60 minutos. &
\textbf{Recursos:} pizarra, notas adhesivas, cuestionario\footnote{\textbf{Cuestionario de Google Forms}: https://forms.gle/62PhEnajWj3m6Mz8A} anónimo.\\
\hline
\textbf{Objetivo:} &
Obtener una primera fotografía del clima de convivencia, identificando conductas que ayudan y que dificultan la vida del grupo.\\
\hline
\textbf{Actividad:} &
Termómetro de convivencia (escala 1 - 10), lluvia de ideas sobre situaciones que mejoran o empeoran el clima y cuestionario anónimo sobre burlas, exclusiones y uso del móvil o redes; se comparten las tendencias generales sin mencionar nombres.\\
\hline
\textbf{Evaluación:} &
El profesorado tutor elabora un breve informe inicial con puntos fuertes y preocupaciones detectadas.\\
\hline
\end{tabularx}

\begin{tabularx}{\textwidth}{|M{0.21\textwidth}|X|}
\hline
\rowcolor{verdePastel}
\multicolumn{2}{|c|}{\textbf{Sesión 2: Diagnóstico individual (tutoría individual).}}\\
\hline
\textbf{Temporalización:} 20--25 minutos por estudiante (4--6 tutorías). &
\textbf{Recursos:} libreta o cuaderno de tutoría.\\
\hline
\textbf{Objetivo:} &
Profundizar en cómo vive cada alumno o alumna el grupo y fijar pequeños objetivos personales de mejora.\\
\hline
\textbf{Actividad:} &
Entrevistas centradas en cómo se siente en clase, qué le preocupa y qué cambios estaría dispuesto/a a intentar, priorizando alumnado que protagoniza o recibe más bromas.\\
\hline
\textbf{Evaluación:} &
Registro de acuerdos y objetivos personales que se retomarán en tutorías posteriores.\\
\hline
\end{tabularx}

\paragraph*{Semana 2 (Sesiones 3, 4 y 5). Impacto de nuestras palabras y de las redes. Apertura a las familias.}

En la segunda semana se trabaja la toma de conciencia sobre el impacto de la comunicación presencial y digital y se comparte el plan con las familias.

\vspace{1cm}

\begin{tabularx}{\textwidth}{|M{0.21\textwidth}|X|}
\hline
\rowcolor{verdePastel}
\multicolumn{2}{|c|}{\textbf{Sesión 3: Palabras que hieren, palabras que ayudan (tutoría grupal).}}\\
\hline
\textbf{Temporalización:} 60 minutos. &
\textbf{Recursos:} tarjetas con frases, mural, rotuladores.\\
\hline
\textbf{Objetivo:} &
Distinguir entre broma compartida y burla que hace daño, y explorar formas de comunicarse más respetuosas.\\
\hline
\textbf{Actividad:} &
Análisis de ejemplos de comentarios habituales, debate sobre sentimientos que generan y reformulación de frases; elaboración de un mural con expresiones que se quieren potenciar y reducir.\\
\hline
\textbf{Evaluación:} &
Participación del alumnado y calidad de las reformulaciones propuestas.\\
\hline
\end{tabularx}

\begin{tabularx}{\textwidth}{|M{0.21\textwidth}|X|}
\hline
\rowcolor{verdePastel}
\multicolumn{2}{|c|}{\textbf{Sesión 4: Lo que subes se queda (tutoría grupal).}}\\
\hline
\textbf{Temporalización:} 60 minutos. &
\textbf{Recursos:} casos ficticios, fotocopias, papel y bolígrafos.\\
\hline
\textbf{Objetivo:} &
Analizar consecuencias de mensajes y contenidos compartidos en redes y en grupos de mensajería.\\
\hline
\textbf{Actividad:} &
Estudio de casos inspirados en situaciones reales, identificación de momentos clave y elaboración de recomendaciones para el uso de grupos de clase y redes sociales.\\
\hline
\textbf{Evaluación:} &
Selección de recomendaciones claras y realistas para integrarlas en el futuro pacto de aula.\\
\hline
\end{tabularx}

\begin{tabularx}{\textwidth}{|M{0.21\textwidth}|X|}
\hline
\rowcolor{verdePastel}
\multicolumn{2}{|c|}{\textbf{Sesión 5: Reunión inicial con familias (tutoría con familias).}}\\
\hline
\textbf{Temporalización:} 60 minutos. &
\textbf{Recursos:} presentación breve, hoja informativa.\\
\hline
\textbf{Objetivo:} &
Presentar el diagnóstico y el plan de actuación, y sensibilizar sobre su papel en la convivencia y el acompañamiento digital.\\
\hline
\textbf{Actividad:} &
Reunión para explicar la situación del grupo, el calendario de trabajo y algunas orientaciones de acompañamiento en el uso de móviles y redes; espacio final para dudas y propuestas.\\
\hline
\textbf{Evaluación:} &
Registro de asistencia e inquietudes comunes para incorporarlas al desarrollo del plan.\\
\hline
\end{tabularx}

\paragraph*{Semana 3 (Sesiones 6, 7 y 8). Entrenamiento en escucha activa y seguimiento con alumnado y familias clave.}

La tercera semana combina una sesión de trabajo grupal centrada en la escucha activa con el seguimiento individualizado del alumnado que fijó objetivos personales y la coordinación con las familias más implicadas.

% =======================
% SESIÓN 6 (GRUPAL) - ESCUCHA ACTIVA
% =======================

\begin{tabularx}{\textwidth}{|M{0.21\textwidth}|X|}
\hline
\rowcolor{verdePastel}
\multicolumn{2}{|c|}{\textbf{Sesión 6: Aprender a escuchar (tutoría grupal).}}\\
\hline
\textbf{Temporalización:} 60 minutos. &
\textbf{Recursos:} fichas con instrucciones, cronómetro, papel y bolígrafos.\\
\hline
\textbf{Objetivos específicos:} &
Desarrollar habilidades básicas de escucha activa e identificar comportamientos verbales y no verbales que ayudan o dificultan sentirse escuchado en una conversación.\\
\hline
\textbf{Actividad:} &
El alumnado trabaja por parejas: una persona habla durante unos minutos sobre un tema cotidiano y la otra escucha sin interrumpir, mostrando atención con la mirada, la postura y pequeños gestos. Después, quien ha escuchado resume lo que ha entendido y se intercambian los roles. Para cerrar, el grupo comparte qué actitudes han favorecido la comunicación y cuáles la han obstaculizado.\\
\hline
\textbf{Evaluación:} &
Cada estudiante realiza una breve autoevaluación sobre su propio nivel de escucha y la tutora recoge observaciones sobre la participación y los avances del grupo en esta habilidad.\\
\hline
\end{tabularx}

% =======================
% SESIÓN 7 (INDIVIDUAL) - SEGUIMIENTO
% =======================

\begin{tabularx}{\textwidth}{|M{0.21\textwidth}|X|}
\hline
\rowcolor{verdePastel}
\multicolumn{2}{|c|}{\textbf{Sesión 7: Tutorías de seguimiento (tutoría individual).}}\\
\hline
\textbf{Temporalización:} 20 - 25 minutos por estudiante. &
\textbf{Recursos:} registros iniciales, libreta de tutoría.\\
\hline
\textbf{Objetivos específicos:} &
Revisar el cumplimiento de los compromisos personales fijados en las primeras entrevistas y ajustar, si es necesario, los objetivos y estrategias de mejora.\\
\hline
\textbf{Actividad:} &
Se realizan tutorías individuales con el alumnado que fijó objetivos en la sesión 2, analizando qué cambios ha intentado introducir, qué dificultades ha encontrado y qué apoyos adicionales puede necesitar para seguir mejorando su forma de relacionarse en el grupo.\\
\hline
\textbf{Evaluación:} &
Se actualizan los registros individuales y se realiza una valoración cualitativa de la evolución de cada estudiante en el aula.\\
\hline
\end{tabularx}

\vspace{2cm}

% =======================
% SESIÓN 8 (FAMILIAS) - ENTREVISTAS CLAVE
% =======================

\begin{tabularx}{\textwidth}{|M{0.21\textwidth}|X|}
\hline
\rowcolor{verdePastel}
\multicolumn{2}{|c|}{\textbf{Sesión 8: Entrevistas con familias clave (tutoría con familias).}}\\
\hline
\textbf{Temporalización:} 15 - 20 minutos por familia. &
\textbf{Recursos:} registros de grupo, notas de tutorías.\\
\hline
\textbf{Objetivos específicos:} &
Compartir el seguimiento realizado con el alumnado más implicado y consensuar estrategias conjuntas de apoyo entre el hogar y el centro.\\
\hline
\textbf{Actividad:} &
Se realizan entrevistas breves con las familias del alumnado más implicado en los conflictos o en las dinámicas de uso de redes. En cada encuentro se contrastan percepciones sobre la convivencia y el comportamiento, se explican los objetivos trabajados en tutoría y se acuerdan pautas coordinadas para reforzar los cambios deseados.\\
\hline
\textbf{Evaluación:} &
Se registran los acuerdos alcanzados con cada familia y, en semanas posteriores, se revisa su impacto observando la evolución del alumnado en el aula y en la convivencia cotidiana.\\
\hline
\end{tabularx}

\paragraph*{Semana 4 (Sesiones 9, 10 y 11). Construcción compartida del pacto de aula.}

En la cuarta semana se elabora el pacto de aula y se concretan compromisos individuales y familiares.

\begin{tabularx}{\textwidth}{|M{0.21\textwidth}|X|}
\hline
\rowcolor{verdePastel}
\multicolumn{2}{|c|}{\textbf{Sesión 9: Construimos nuestro pacto de aula (tutoría grupal).}}\\
\hline
\textbf{Temporalización:} 60 minutos. &
\textbf{Recursos:} pizarra, notas adhesivas, borrador de propuestas.\\
\hline
\textbf{Objetivo:} &
Acordar normas y compromisos sobre comunicación y uso de TIC asumidos por el grupo.\\
\hline
\textbf{Actividad:} &
Recogida y agrupación de propuestas del alumnado, debate de puntos de desacuerdo y redacción conjunta del pacto; validación mediante votación.\\
\hline
\textbf{Evaluación:} &
Grado de consenso, claridad y realismo del pacto aprobado.\\
\hline
\end{tabularx}

\begin{tabularx}{\textwidth}{|M{0.21\textwidth}|X|}
\hline
\rowcolor{verdePastel}
\multicolumn{2}{|c|}{\textbf{Sesión 10: Compromisos personales ante el pacto (tutoría individual).}}\\
\hline
\textbf{Temporalización:} 15--20 minutos por estudiante. &
\textbf{Recursos:} texto del pacto, libreta de tutoría.\\
\hline
\textbf{Objetivo:} &
Concretar uno o dos compromisos personales vinculados al pacto y detectar posibles resistencias.\\
\hline
\textbf{Actividad:} &
Revisión del pacto con el alumnado más implicado y formulación de compromisos realistas (por ejemplo, cambiar cierto tipo de comentarios o evitar reenviar determinados contenidos).\\
\hline
\textbf{Evaluación:} &
Registro de compromisos e indicadores sencillos para su seguimiento.\\
\hline
\end{tabularx}

\begin{tabularx}{\textwidth}{|M{0.21\textwidth}|X|}
\hline
\rowcolor{verdePastel}
\multicolumn{2}{|c|}{\textbf{Sesión 11: Comunicación del pacto a las familias (tutoría con familias).}}\\
\hline
\textbf{Temporalización:} 45--60 minutos. &
\textbf{Recursos:} copia del pacto, circular o presentación.\\
\hline
\textbf{Objetivo:} &
Dar a conocer el pacto a las familias y pedir su colaboración para reforzarlo desde casa.\\
\hline
\textbf{Actividad:} &
Reunión presencial o virtual, o envío de circular, explicando el proceso seguido y proponiendo maneras concretas de apoyar los acuerdos en el ámbito familiar.\\
\hline
\textbf{Evaluación:} &
Control de familias informadas y análisis de sus comentarios para futuras mejoras.\\
\hline
\end{tabularx}

\paragraph*{Semana 5 (Sesiones 12 y 13). Evaluación y cierre.}

La última semana se dedica a valorar el impacto global de la intervención y a consolidar compromisos de continuidad.

\begin{tabularx}{\textwidth}{|M{0.21\textwidth}|X|}
\hline
\rowcolor{verdePastel}
\multicolumn{2}{|c|}{\textbf{Sesión 12: Cierre individual con alumnado implicado (tutoría individual).}}\\
\hline
\textbf{Temporalización:} 20--25 minutos por estudiante. &
\textbf{Recursos:} registros iniciales y de seguimiento.\\
\hline
\textbf{Objetivo:} &
Valorar, junto a cada alumno o alumna, los cambios producidos y fijar nuevas metas si procede.\\
\hline
\textbf{Actividad:} &
Comparación entre la situación inicial y la actual, identificación de avances y dificultades, y planteamiento de posibles pasos posteriores.\\
\hline
\textbf{Evaluación:} &
Síntesis de la evolución de cada estudiante para el informe final de grupo.\\
\hline
\end{tabularx}

\begin{tabularx}{\textwidth}{|M{0.21\textwidth}|X|}
\hline
\rowcolor{verdePastel}
\multicolumn{2}{|c|}{\textbf{Sesión 13: Reunión final con familias (tutoría con familias).}}\\
\hline
\textbf{Temporalización:} 60 minutos. &
\textbf{Recursos:} síntesis del proceso, acta de reunión.\\
\hline
\textbf{Objetivo:} &
Compartir los resultados principales del plan y acordar posibles líneas de continuidad en la colaboración centro-familia.\\
\hline
\textbf{Actividad:} &
Presentación de avances y aspectos en proceso, diálogo con las familias y recogida de propuestas para mantener o reforzar ciertas medidas.\\
\hline
\textbf{Evaluación:} &
Acta con aportaciones y sugerencias para futuras actuaciones en el centro.\\
\hline
\end{tabularx}

\section{Tercera parte: Conclusiones de la tarea: reflexión sobre los aprendizajes y prospectiva del proceso llevado a cabo}\label{sec:parte-3}

\subsection{¿Qué significa ser tutor y cómo lo integro en mi modo de entender la enseñanza o la profesión docente?}

A lo largo de esta PEC he ido reformulando bastante qué entiendo por ser tutor. Si al principio asociaba la tutoría sobre todo a ciertas funciones organizativas (reuniones con familias, control de faltas, transmisión de información del centro), ahora la percibo como una responsabilidad nuclear de la profesión docente.

Ser tutor, para mí, significa ser una figura de referencia que acompaña al grupo y a cada estudiante en tres planos que se entrelazan: académico, personal y relacional. Implica conocer la historia y el momento vital del alumnado, estar atento a los pequeños signos de malestar, facilitar espacios de escucha y ayudar a que la vida del grupo sea habitable. Supone también ejercer de mediador entre los distintos agentes (equipo docente, familias, orientación, jefatura de estudios) para que las decisiones que se toman sobre los estudiantes tengan sentido y se coordinen.

Esta visión encaja con un modo de entender la enseñanza donde el aprendizaje no se reduce a ``explicar bien una asignatura'', sino a crear condiciones para que cada alumno pueda implicarse, sentirse reconocido y aprender con otros. En ese marco, la tutoría no es un añadido ni un ``hueco'' en el horario, sino el lugar privilegiado para trabajar la convivencia, la inclusión y la construcción de la identidad académica y personal del alumnado. Integrar la función tutorial en mi identidad profesional significa asumir que parte de mi tarea como docente es sostener estos procesos y no delegarlos por completo en otros perfiles del centro.

Además, el hecho de estar ya vinculado al sistema educativo me ayuda a aterrizar esta idea en experiencias concretas: he podido ver de cerca cómo cambia el clima de un grupo cuando hay una tutoría activa y coordinada, frente a cuando la tutoría se limita a trámites administrativos. Esto refuerza mi convicción de que el rol de tutor tiene un impacto real en la experiencia escolar de los estudiantes.

\subsection{¿Cómo proyecto mi futura práctica tutorial con relación a la mejora de la convivencia en el aula?}

A partir del trabajo realizado, me resulta difícil imaginar una práctica tutorial que no esté claramente orientada a la convivencia. Entiendo la convivencia no como ausencia de conflictos, sino como la capacidad del grupo para gestionarlos de forma justa, dialogada e inclusiva. Desde esta perspectiva, la tutoría debería tener un enfoque principalmente preventivo y formativo, más que reactivo.

En mi futura práctica, me gustaría estructurar la tutoría como un espacio estable donde el alumnado sepa que se puede hablar de lo que pasa en el grupo: tensiones, malentendidos, comentarios que hacen daño, uso de redes sociales, sentimientos de exclusión, etc. Eso implica programar actividades específicas (dinámicas de cohesión, trabajo sobre emociones, comunicación asertiva, análisis de situaciones de acoso o microagresiones) y combinar momentos más guiados con espacios de palabra abierta.

También proyecto la tutoría como un puente entre lo que ocurre en el aula y el resto de la vida del centro. Trabajar la convivencia desde la tutoría supone coordinarse con el Plan de Convivencia, con otros tutores y con el Departamento de Orientación, de forma que las actuaciones no dependan solo de la buena voluntad de una persona, sino que se inscriban en una línea de centro. En este sentido, me imagino como un tutor que no solo ``aplica actividades'', sino que participa en la reflexión colectiva sobre cómo mejorar el clima escolar.

Finalmente, quiero que mi práctica tutorial tenga en cuenta la diversidad real del alumnado: trayectorias académicas diferentes, contextos familiares muy variados, diversidad cultural, de género, de orientación, etc. La convivencia se juega precisamente en cómo se acoge y se gestiona esa diversidad. La tutoría debería ayudar a que cada alumno se sienta parte del grupo sin tener que renunciar a quién es.

\subsection{Coherencia del plan, dificultades previsibles y papel del PAT}

Al revisar el plan de actuación que he elaborado en la segunda parte de la tarea, considero que la secuencia necesidad - objetivos - actuaciones mantiene una coherencia razonable. Parto de una necesidad clara relacionada con la convivencia (conflictos relacionales, comentarios excluyentes y un clima de aula mejorable) y los objetivos se centran en hacer visible el problema, dotar al grupo de herramientas para relacionarse mejor, reforzar la posición de los alumnos más vulnerables y coordinar las respuestas del centro. Las actividades diseñadas (trabajo grupal, espacios de reflexión, alguna intervención individual y la implicación de las familias) buscan responder directamente a esos objetivos, y la temporalización en 4 - 6 semanas obliga a priorizar aquello que puede tener más impacto en poco tiempo.

Aun así, soy consciente de varias dificultades. La primera es la resistencia al cambio, tanto por parte de algunos estudiantes como, a veces, de parte del profesorado. Cambiar dinámicas instaladas, bromas normalizadas o formas de relacionarse que llevan tiempo funcionando no es sencillo, y puede aparecer rechazo o trivialización (``otra charla más sobre lo de siempre''). La segunda dificultad tiene que ver con la participación de las familias: no siempre es fácil que se impliquen, bien por horarios, por desconfianza o porque no perciben el problema como algo grave. Una tercera dificultad es estructural: el tiempo limitado, la presión del currículo y la tendencia a relegar la tutoría frente a las materias ``troncales''.

Para anticiparme a estas dificultades, veo necesario cuidar varios aspectos. Con el alumnado, explicar con transparencia el sentido de las actividades, implicarles en el diagnóstico y en la búsqueda de soluciones, y procurar que las propuestas sean dinámicas, participativas y conectadas con su realidad (incluyendo, por ejemplo, el impacto de las redes o los grupos de WhatsApp en los conflictos). Con las familias, ofrecer canales diversos (presencial, online, comunicación escrita), mostrar ejemplos concretos de situaciones que se quieren prevenir y dejar claro que no se busca culpabilizar, sino construir una alianza educativa. Con el resto del profesorado, compartir el plan, escuchar sus percepciones sobre el grupo y buscar pequeñas acciones coordinadas que acompañen lo que se trabaja en la tutoría.

En este punto, el Plan de Acción Tutorial (PAT) cobra una relevancia central. Si el plan que he diseñado se integra en el PAT, deja de ser una intervención aislada y se convierte en parte de una estrategia más amplia de centro. El PAT permite dar continuidad a lo iniciado en esas 5 semanas (por ejemplo, retomando algunos temas más adelante, conectándolos con proyectos de aula o incorporándolos al plan de convivencia) y evita que la tutoría dependa exclusivamente de la iniciativa individual del tutor. También ayuda a asegurar que otros grupos trabajen líneas similares, reforzando así una cultura de centro en torno a la convivencia y la inclusión.

\subsection{Aprendizajes de la tarea y su contribución a mi formación como futuro docente}

Esta tarea me ha enseñado, en primer lugar, a mirar la tutoría con más profundidad. He pasado de una visión algo difusa a identificar con claridad distintas dimensiones de la función tutorial (académica, personal, vocacional, social) y a comprender mejor cómo se articulan con la convivencia y la inclusión. También he tomado conciencia de la importancia del marco normativo y de los documentos de centro (Proyecto Educativo de Centro, Plan de Convivencia, PAT), no como meros requisitos burocráticos, sino como herramientas que pueden dar sentido y continuidad a las actuaciones.

En segundo lugar, el ejercicio de diseñar un plan concreto me ha obligado a bajar de la teoría a la práctica: definir una necesidad, acotarla, formular objetivos realistas para un tiempo limitado y pensar actividades que sean viables en un contexto realista de secundaria. Esto me ha hecho ver que no basta con ``tener buenas ideas''; hay que priorizar, negociar tiempos, prever resistencias y pensar qué indicadores me permitirían saber si algo está funcionando.

En cuanto a la conexión con otras asignaturas, esta tarea dialoga claramente con lo trabajado en materias como Sociedad, Familia y Educación, donde se analizan las relaciones entre escuela y contexto social y el papel de las familias; con la asignatura de Desarrollo Psicológico y Aprendizaje, que ayuda a comprender mejor la etapa adolescente, los procesos de desarrollo y las dinámicas de grupo; y con el Prácticum, donde se observan tutorías reales y se contrastan las propuestas teóricas con las prácticas de los centros.

Desde el punto de vista de mi formación como futuro docente, esta tarea ha sido especialmente valiosa porque me ha ayudado a construir una imagen más completa de lo que implica el trabajo docente en secundaria. No se trata solo de dominar un contenido y saber explicarlo, sino de asumir la responsabilidad de acompañar procesos personales y grupales complejos. Me llevo la idea de que la tutoría es un espacio privilegiado para hacer visible la dimensión ética y relacional de la educación, y que requiere tiempo, formación específica y apoyo institucional.

En definitiva, la PEC me ha permitido articular mis experiencias previas, las aportaciones teóricas del curso y un primer ejercicio de planificación de la acción tutorial. Salgo de este proceso con más preguntas que respuestas cerradas, pero con la sensación de que dispongo de un marco mucho más sólido para pensar mi futura práctica tutorial y su contribución a la convivencia y la inclusión en los centros de secundaria.

\printbibliography

\newpage

\section{Declaración de Autoría de la PEC}

\begin{flushright}
Fecha: \today
\end{flushright}

\vspace{0.3cm}

\noindent\textbf{Quién suscribe:}

\vspace{0.3cm}

\noindent
\begin{tabularx}{\textwidth}{|>{\bfseries\raggedright\arraybackslash}p{4.7cm}|X|}
\hline
ESTUDIANTE (apellidos y nombre): & Dustin Randel de Lara García \\ \hline
D.N.I./Pasaporte: & 43216308K \\ \hline
Curso académico: & 2025-2026 \\ \hline
Convocatoria (ordinaria/extraordinaria): & Ordinaria \\ \hline
ASIGNATURA & Procesos y Contextos Educativos \\ \hline
\end{tabularx}

\vspace{0.3cm}

\noindent Es el/la autor/a del trabajo (Prueba de Evaluación Continua, PEC), que lleva por título:

\vspace{0.3cm}

\noindent
\begin{tabularx}{\textwidth}{|X|}
\hline
\textbf{Tutoría como promotora de la convivencia en educación secundaria} \\ 
\hline
\end{tabularx}

\vspace{0.3cm}

\noindent\textbf{Manifiesta:}

\vspace{0.3cm}

\begin{itemize}
  \item Que el trabajo remitido es un documento original elaborado individualmente.
  \item Que las aportaciones intelectuales de otros autores consideradas en el trabajo se han referenciado debidamente.
  \item Que, para la elaboración del presente trabajo, no se han utilizado herramientas de Inteligencia Artificial Generativa (IAG) con el propósito de hacer pasar como propio cualquier pasaje de texto producido con esta tecnología.
  \item Que, si se demostrara lo contrario, el abajo firmante aceptará las medidas disciplinarias o sancionadoras que correspondan.
\end{itemize}

\vspace{0.3cm}

\begin{center}
Nombre, apellidos y firma del/la estudiante:
\end{center}

\begin{center}
Dustin Randel de Lara García
\end{center}

\end{spacing}

\end{document}
