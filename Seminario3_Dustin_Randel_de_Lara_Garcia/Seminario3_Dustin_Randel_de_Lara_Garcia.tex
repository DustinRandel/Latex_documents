\documentclass[11pt,a4paper]{article}

% ---- Márgenes
\usepackage[a4paper, left=2cm, right=2cm, top=2.5cm, bottom=2.5cm]{geometry}

% ---- Idioma y fuentes
\usepackage[utf8]{inputenc}
\usepackage[T1]{fontenc}
\usepackage[spanish, es-tabla]{babel}
\usepackage{lmodern}

% ---- Matemáticas y tablas
\usepackage{amsmath,amsfonts,amssymb}
\usepackage{tabularx,booktabs}

% ---- Gráficos
\usepackage{graphicx}
\usepackage{subcaption}
\usepackage{float}
\usepackage[table,xcdraw]{xcolor}
\definecolor{verdePastel}{RGB}{204, 235, 197}
\definecolor{verdeFresco}{RGB}{60,160,60}

% ---- Estilo y utilidades
\usepackage{fancyhdr}
\setlength{\headheight}{14pt}
\usepackage{csquotes}
\usepackage{parskip}   % separa párrafos sin sangría
\usepackage{array,longtable}
\renewcommand{\arraystretch}{1.4}
\newcolumntype{L}[1]{>{\raggedright\arraybackslash}p{#1}}
\newcolumntype{C}[1]{>{\centering\arraybackslash}p{#1}}
\newcolumntype{B}[1]{>{\raggedright\arraybackslash}p{#1}}

% ---- Bibliografía (mejor con biber)
\usepackage[
  backend=biber,        % o 'bibtex' si no puedes usar biber
  style=numeric,        % números [1], [2], ...
  sorting=none,         % <- orden según aparición en el texto
  maxbibnames=99
]{biblatex}

\addbibresource{referencias.bib}

% ---- Hipervínculos (cargar al final)
\usepackage{hyperref}
\hypersetup{
  pdfauthor={Dustin Randel De Lara García},
  pdftitle={Seminario III - Cómo aprenden y cómo son. Concepciones y creencias de los docentes sobre los alumnos},
  pdfkeywords={Seminario III, Máster en Formación del Profesorado, UNED, Educación},
  colorlinks=true,
  linkcolor=verdeFresco,
  urlcolor=verdeFresco,
  citecolor=verdeFresco
}

% ---- “Tabla” en lugar de “Cuadro” (por si acaso)
\addto\captionsspanish{\renewcommand{\tablename}{Tabla}}

% ---- Encabezados/pies
\pagestyle{fancy}
\rhead{UNED}
\lhead{Seminario III - Cómo aprenden y cómo son. Concepciones y creencias sobre los alumnos} 
\cfoot{\thepage}

\date{\today}

\begin{document}

% ---- Portada
\begin{center}
  {\includegraphics[width=0.5\textwidth]{icono_uned.jpg}\par}
  \vspace{1cm}
  {\bfseries\LARGE Universidad Nacional de Educación a Distancia \par}
  \vspace{1cm}
  {\scshape\Large Facultad de Educación\par}
  \vspace{0.5cm}
  {\scshape\Large Máster en Formación del Profesorado\par}
  \vspace{0.5cm}
  {\scshape\Large Prácticum \par}
  \vspace{1.5cm}
  {\scshape\Huge Cómo aprenden y cómo son. Concepciones y creencias de los docentes sobre los alumnos \par}
  \vspace{1.5cm}
  {\itshape\Large Seminario III \par}
  \vfill
  {\Large Dustin Randel De Lara García \par}
  \vfill
  {\Large \today \par}
  \thispagestyle{empty}
\end{center}

% ---- Índice
\newpage
\tableofcontents
\newpage

\section{Actividad 1}\label{sec:actividad1}
\subsection{Sobre bilingüismo}
\begin{enumerate}
  \item "Una segunda lengua hay que aprenderla cuando se haya aprendido bien la lengua materna"
  \item "Es beneficioso aprender dos lenguas simultáneamente"
\end{enumerate}
 
\section{Actividad 4}\label{sec:actividad2}

\newpage
\printbibliography
\end{document}
