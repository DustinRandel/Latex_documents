\documentclass[11pt,a4paper]{article}

% ---- Márgenes
\usepackage[a4paper, left=2cm, right=2cm, top=2.5cm, bottom=2.5cm]{geometry}

% ---- Idioma y fuentes
\usepackage[utf8]{inputenc}
\usepackage[T1]{fontenc}
\usepackage[spanish, es-tabla]{babel}
\usepackage{lmodern}

% ---- Matemáticas y tablas
\usepackage{amsmath,amsfonts,amssymb}
\usepackage{tabularx,booktabs}

% ---- Gráficos
\usepackage{graphicx}
\usepackage{subcaption}
\usepackage{float}
\usepackage[table,xcdraw]{xcolor}
\definecolor{verdePastel}{RGB}{204, 235, 197}
\definecolor{verdeFresco}{RGB}{60,160,60}

% ---- Estilo y utilidades
\usepackage{fancyhdr}
\setlength{\headheight}{14pt}
\usepackage{csquotes}
\usepackage{parskip}   % separa párrafos sin sangría
\usepackage{array,longtable}
\renewcommand{\arraystretch}{1.4}
\newcolumntype{L}[1]{>{\raggedright\arraybackslash}p{#1}}
\newcolumntype{C}[1]{>{\centering\arraybackslash}p{#1}}
\newcolumntype{B}[1]{>{\raggedright\arraybackslash}p{#1}}

% ---- Bibliografía (mejor con biber)
\usepackage[
  backend=biber,        % o 'bibtex' si no puedes usar biber
  style=numeric,        % números [1], [2], ...
  sorting=none,         % <- orden según aparición en el texto
  maxbibnames=99
]{biblatex}

\addbibresource{referencias.bib}

% ---- Hipervínculos (cargar al final)
\usepackage{hyperref}
\hypersetup{
  pdfauthor={Dustin Randel De Lara García},
  pdftitle={Formación Permanente del Profesorado en Secundaria y FP: El Programa Erasmus+ como Vía de Desarrollo Profesional},
  pdfkeywords={formación permanente, profesorado, Erasmus+, desarrollo profesional, educación secundaria, formación profesional},
  colorlinks=true,
  linkcolor=verdeFresco,
  urlcolor=verdeFresco,
  citecolor=verdeFresco
}

% ---- “Tabla” en lugar de “Cuadro” (por si acaso)
\addto\captionsspanish{\renewcommand{\tablename}{Tabla}}

% ---- Encabezados/pies
\pagestyle{fancy}
\rhead{UNED}
\lhead{Seminario II - Formación permanente del profesorado de secundaria/FP: El caso Erasmus+} 
\cfoot{\thepage}

\date{\today}

\begin{document}

% ---- Portada
\begin{center}
  \includegraphics[width=0.32\textwidth]{icono_uned.jpg}\par
  \vspace{0.3cm}
  {\bfseries\large Universidad Nacional de Educación a Distancia \par}
  {\scshape Facultad de Educación – Máster en Formación del Profesorado\par}
  \vspace{0.4cm}
  {\scshape Prácticum – Seminario II - Illes Ballears\par}
  \vspace{0.4cm}
  {\itshape\large Formación permanente del profesorado de secundaria/FP: El caso Erasmus+\par}
  \vspace{0.5cm}
  {\large Dustin Randel De Lara García, 43216308K \par}
  {\large \today \par}
\end{center}

\vspace{0.5cm}
\hrule

\renewcommand{\contentsname}{Índice de contenidos}
\tableofcontents

\section*{Proyecto Erasmus+ KA122-SCH «Watering our Future» (IES Isla Verde, Algeciras): 2021-1-ES01-KA122-SCH-000015629}

En este seminario analizo un proyecto real de Erasmus+ (KA1) vinculado a una feria de ciencia para entender cómo estas experiencias pueden mejorar mi futura práctica docente en Física y Química.

El proyecto elegido es el «Mójate por la educación / Watering our Future», del IES Isla Verde (Algeciras)\parencite{iesislaverde_k122_2021,sepie_ka122_2021}, centrado en sostenibilidad del agua, innovación metodológica e idiomas.

Mi objetivo es describirlo y justificar su relevancia para mi formación permanente a la luz de las prioridades de Erasmus+\parencite{ec_guide_2022,ec_keyaction1_2025}.
\section{Características del proyecto}\label{sec:caracteristicas-del-proyecto}

La solicitud del proyecto organiza sus líneas de actuación en cinco ejes principales:

\begin{itemize}
  \item \textbf{Competencia digital}. Integración de herramientas TIC para crear materiales interactivos y gestionar actividades en línea.
  \item \textbf{Metodologías innovadoras e inclusivas}. Uso de \emph{blended learning}, \emph{flipped classroom} y aprendizaje basado en proyectos cooperativos para atender mejor a la diversidad.
  \item \textbf{Dimensión europea y lenguas extranjeras}. Mayor presencia de ingléscomo lengua de trabajo y fortalecimiento de la internacionalización mediante cooperación con centros socios.
  \item \textbf{Apertura a otros modelos educativos}. Observación de prácticas docentes en Europa para transferirlas y adaptarlas al contexto del proyecto.
  \item \textbf{Sostenibilidad}. Trabajo en torno al agua como eje temático, en coherencia con la Agenda 2030 y con iniciativas ambientales ya presentes en el centro\parencite{islaVerde_KA122_application_2021}.
\end{itemize}

Como complemento, se desarrolla un proyecto eTwinning para dar continuidad al trabajo multinacional fuera de las movilidades físicas\parencite{islaVerde_etwinning_waterfuture_2022}.

\subsection*{Actividades de movilidad y formación}

El plan combina formación del profesorado, estancias de observación y movilidades de alumnado\parencite{islaVerde_KA122_application_2021}:

\begin{itemize}
  \item \textbf{Cursos estructurados}. Formación en competencias digitales, metodologías activas e inclusión, con compromiso de transferencia interna al resto del claustro.
  
  \item \textbf{Estancias de \emph{job shadowing}}. El profesorado observa prácticas de enseñanza de ciencias, organización de laboratorios, atención a la diversidad y proyectos ambientales en tres centros europeos:
  \begin{itemize}
      \item Liceo Scientifico Cocito (Italia)\parencite{islaVerde_jobshadowing_cocito_2022}.
      \item Slezské Gymnázium (República Checa)\parencite{islaVerde_recepcion_opava_2022}.
      \item Gymnasium Västerhöjd (Suecia)\parencite{islaVerde_jobshadowing_skovde_2022}.
  \end{itemize}
  Estas observaciones se complementan con la visita de los socios al IES Isla Verde, donde se presentan proyectos ambientales propios como el ecohuerto y actividades del programa Aldea\parencite{islaVerde_KA122_application_2021}.

  \item \textbf{Movilidades de alumnado}. Estudiantes de 4.º ESO trabajan conjuntamente en actividades sobre sostenibilidad del agua y divulgación científica. Un ejemplo es la movilidad con el Slezské Gymnázium (Opava), que incluye:
  \begin{itemize}
      \item Convivencia con familias del centro.
      \item Visita a la planta potabilizadora del río de la Miel.
      \item Actividades conjuntas de aula y culturales en el Museo Municipal.
      \item Jornada de trabajo en la feria de ciencia Diverciencia\parencite{islaVerde_recepcion_opava_2022}.
  \end{itemize}
\end{itemize}

\subsection*{Relación con la feria de ciencia Diverciencia}

Diverciencia es una feria internacional de ciencia en la calle que reúne anualmente cerca de un centenar de stands preparados por alumnado divulgador\parencite{diverciencia_descubre}. Dentro del proyecto KA122, la feria se convierte en un espacio clave de aplicación y difusión: los grupos participantes presentan proyectos científicos sobre agua y sostenibilidad, realizan experimentos adaptados al formato divulgativo y elaboran carteles o materiales bilingües\parencite{islaVerde_recepcion_opava_2022}. Para el profesorado visitante, Diverciencia funciona también como un escenario de \emph{job shadowing}, al permitir observar cómo se integran la sostenibilidad, la experimentación y la comunicación científica en un evento abierto a toda la comunidad educativa\parencite{islaVerde_recepcion_opava_2022}.

\section{¿Por qué es interesante este proyecto?}\label{sec:porque-es-interesante}

El proyecto me resulta especialmente interesante porque conecta directamente con los objetivos de Erasmus+ y con una visión exigente de la formación permanente docente.

Por un lado, encaja plenamente con las prioridades de KA1: la movilidad se entiende como un medio para mejorar el centro, en coherencia con las directrices europeas\parencite{ec_guide_2022,ec_keyaction1_2025}. En el IES Isla Verde, las estancias formativas y el job shadowing forman parte de un plan de mejora centrado en digitalización, inclusión y sostenibilidad\parencite{iesislaverde_k122_2021}.

Además, el eje temático del agua se vincula de forma natural con el currículo de Física y Química. Permite abordar contenidos como energía hidráulica, mecánica de fluidos, contaminación o tratamiento del agua desde un enfoque contextualizado, alineado con la Agenda 2030\parencite{ec_guide_2022}.

A ello se suma un aspecto que considero clave: la participación en Diverciencia. Esta feria impulsa la comunicación pública de la ciencia, lo que exige al alumnado diseñar experimentos, crear materiales explicativos y presentarlos a públicos diversos\parencite{diverciencia_web,diverciencia_descubre}.

Finalmente, valoro que el proyecto se haya desarrollado en un instituto ordinario, con un contexto fácilmente replicable\parencite{iesislaverde_k122_2021}. Esto me permite imaginar cómo aplicar una iniciativa similar en mi futuro centro, adaptando el eje temático y la participación en ferias científicas.

\subsection{Aportación a la formación permanente del profesorado de secundaria}\label{subsec:aportacion-a-la-formacion-permanente-del-profesorado-de-secundaria}

Desde la perspectiva de la formación permanente, este proyecto muestra que el desarrollo profesional no consiste en acumular cursos, sino en impulsar cambios sostenidos en la cultura pedagógica del centro. Las movilidades se orientan a digitalización, inclusión y sostenibilidad, con planes de mejora y compromisos reales de transferencia interna\parencite{ec_guide_2022,iesislaverde_k122_2021}. Participar en un proyecto así implicaría:

\begin{itemize}
\item Contrastar la práctica docente con la de otros centros europeos y observar cómo gestionan el laboratorio, la diversidad o la sostenibilidad en sus proyectos científicos.
\item Recibir formación en metodologías útiles para Física y Química (ABP, CLIL) y asumir la responsabilidad de adaptarlas y compartirlas en el departamento.
\item Desarrollar competencias transversales (trabajo en equipo, comunicación intercultural y gestión de proyectos) necesarias para impulsar innovación real.
\end{itemize}

En definitiva, el proyecto no solo actualiza la didáctica, sino que sitúa al profesorado en un rol activo dentro del centro, liderando la difusión y aplicación de lo aprendido.

\section{Transferencia del aprendizaje a la práctica docente cotidiana}\label{sec:Transferencia-aprendizaje}

Entiendo la transferencia del aprendizaje como el proceso por el cual lo vivido en un proyecto Erasmus+ pasa a la práctica diaria, mejorando tanto la experiencia del alumnado como la cultura profesional del centro. En mi caso, esta transferencia se concretaría en varias líneas.

En primer lugar, un uso más sistemático del ABP en Física y Química. Diseñar unidades en torno a un producto final (por ejemplo, un stand sobre calidad del agua) encaja con la filosofía del proyecto y con los objetivos de Diverciencia\parencite{diverciencia_web,diverciencia_descubre}. Algunos temas podrían convertirse en miniproyectos con exposiciones públicas.

En segundo lugar, un trabajo cooperativo más estructurado. Observar buenas prácticas en los centros socios ayudaría a reorganizar el laboratorio mediante roles, tareas multinivel y agrupamientos heterogéneos que garanticen la participación de todo el alumnado.

En tercer lugar, una integración real de la competencia digital. La formación en TIC (simulaciones, laboratorios virtuales, hojas de cálculo, plataformas colaborativas) se traduciría en su uso habitual tanto en la investigación experimental como en la creación de productos de divulgación.

Por último, incorporaría pequeñas dosis de dimensión europea y lingüística: glosarios en inglés, lecturas guiadas o la presentación parcial de stands en lengua extranjera, sin convertir la materia en bilingüe, pero aprovechando la dimensión internacional del proyecto.
\section{Aplicación de las habilidades y conocimientos en mi día a día}\label{sec:aplicacion-dia-a-dia}

Si participara en un proyecto como «Watering our Future», los aprendizajes adquiridos se reflejarían en decisiones concretas de planificación, organización y evaluación.

En la planificación, incluiría cada curso al menos un proyecto con producto final divulgativo sobre sostenibilidad (agua, energía, cambio climático), pensado para una audiencia más amplia que la clase y orientado a participar en ferias científicas siguiendo el modelo de Diverciencia\parencite{diverciencia_web,diverciencia_descubre}.

En la organización del aula, aplicaría estrategias de trabajo cooperativo e inclusión: grupos heterogéneos, roles claros, materiales graduados y estaciones de aprendizaje que permitan que todo el alumnado participe y progrese en el laboratorio.

En la evaluación, utilizaría instrumentos que valoren tanto los contenidos como las competencias científicas y comunicativas: rúbricas para experimentos y presentaciones, autoevaluaciones, coevaluaciones y momentos de reflexión sobre el proceso.

Además, mantendría la dimensión europea mediante pequeñas actividades de conexión con otros centros (como proyectos eTwinning), el uso puntual de recursos en inglés y la comparación de datos científicos locales con los de otros países\parencite{ec_keyaction1_2025,iesislaverde_k122_2021}.

\section{Supuestos de aplicación de las competencias Erasmus+}\label{sec:posibles-contextos-de-aplicación}

\subsection*{Supuesto 1: Proyecto «La huella hídrica del centro» (3.º ESO)}

Proyecto interdisciplinar en el que el alumnado calcula la huella hídrica del instituto a partir de mediciones reales. Tras recorrer distintos espacios (aulas, laboratorios, aseos, patios), recogerían datos, estimarían volúmenes y elaborarían gráficas con herramientas digitales, además de analizar parámetros químicos básicos como el pH.

El producto final sería un stand divulgativo para la feria científica del centro o para un evento tipo Diverciencia\parencite{diverciencia_descubre}, con maquetas, infografías y pequeños experimentos.

\subsection*{Supuesto 2: Stand «Energía del agua» (4.º ESO – 1.º Bachillerato)}

A partir de lo observado en las estancias de job shadowing en Italia y la República Checa\parencite{islaVerde_jobshadowing_cocito_2022,islaVerde_recepcion_opava_2022}, el alumnado diseñaría pequeños dispositivos basados en energía hidráulica (ruedas de agua, turbinas caseras o mini generadores). La actividad permite trabajar conceptos de energía y rendimiento a través de experiencias prácticas.

Parte del material divulgativo se elaboraría en inglés. El stand final serviría para explicar los experimentos al público.

\subsection*{Supuesto 3: Laboratorio inclusivo con estaciones de aprendizaje (2.º ESO)}

Inspirado en prácticas observadas en los centros socios durante las estancias de \emph{job shadowing}, podría organizar el laboratorio mediante estaciones de aprendizaje secuenciadas y multinivel\parencite{islaVerde_jobshadowing_skovde_2022}. Cada estación abordaría un concepto básico (cambios de estado, reacciones simples, \dots) con tareas adaptadas a distintos niveles de complejidad.

El trabajo se organizaría en grupos heterogéneos con roles rotatorios (coordinación, manipulación, registro y seguridad), garantizando la participación activa de todo el alumnado.

Al finalizar, cada grupo presentaría un pequeño póster o vídeo sobre una de las estaciones, generando un producto final que podría mostrarse al curso siguiente o en una feria de ciencia interna.

\subsection*{Supuesto 4: Mini proyecto eTwinning sobre calidad del agua}

Basado en el eTwinning asociado al proyecto\parencite{islaVerde_etwinning_waterfuture_2022}, se realizaría un pequeño estudio comparado de calidad del agua con un centro europeo. El alumnado mediría parámetros como el pH, compartiría los resultados en una plataforma digital y elaboraría conclusiones conjuntas.

Este mini proyecto complementa el Supuesto 1 y permite generar una presentación o póster bilingüe para una feria científica presencial o virtual.

\printbibliography

\end{document}
