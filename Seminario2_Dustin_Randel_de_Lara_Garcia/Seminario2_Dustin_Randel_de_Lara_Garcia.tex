\documentclass[11pt,a4paper]{article}

% ---- Márgenes
\usepackage[a4paper, left=2cm, right=2cm, top=2.5cm, bottom=2.5cm]{geometry}

% ---- Idioma y fuentes
\usepackage[utf8]{inputenc}
\usepackage[T1]{fontenc}
\usepackage[spanish, es-tabla]{babel}
\usepackage{lmodern}

% ---- Matemáticas y tablas
\usepackage{amsmath,amsfonts,amssymb}
\usepackage{tabularx,booktabs}

% ---- Gráficos
\usepackage{graphicx}
\usepackage{subcaption}
\usepackage{float}
\usepackage[table,xcdraw]{xcolor}
\definecolor{verdePastel}{RGB}{204, 235, 197}
\definecolor{verdeFresco}{RGB}{60,160,60}

% ---- Estilo y utilidades
\usepackage{fancyhdr}
\setlength{\headheight}{14pt}
\usepackage{csquotes}
\usepackage{parskip}   % separa párrafos sin sangría
\usepackage{array,longtable}
\renewcommand{\arraystretch}{1.4}
\newcolumntype{L}[1]{>{\raggedright\arraybackslash}p{#1}}
\newcolumntype{C}[1]{>{\centering\arraybackslash}p{#1}}
\newcolumntype{B}[1]{>{\raggedright\arraybackslash}p{#1}}

% ---- Bibliografía (mejor con biber)
\usepackage[
  backend=biber,        % o 'bibtex' si no puedes usar biber
  style=numeric,        % números [1], [2], ...
  sorting=none,         % <- orden según aparición en el texto
  maxbibnames=99
]{biblatex}

\addbibresource{referencias.bib}

% ---- Hipervínculos (cargar al final)
\usepackage{hyperref}
\hypersetup{
  pdfauthor={Dustin Randel De Lara García},
  pdftitle={Formación Permanente del Profesorado en Secundaria y FP: El Programa Erasmus+ como Vía de Desarrollo Profesional},
  pdfkeywords={formación permanente, profesorado, Erasmus+, desarrollo profesional, educación secundaria, formación profesional},
  colorlinks=true,
  linkcolor=verdeFresco,
  urlcolor=verdeFresco,
  citecolor=verdeFresco
}

% ---- “Tabla” en lugar de “Cuadro” (por si acaso)
\addto\captionsspanish{\renewcommand{\tablename}{Tabla}}

% ---- Encabezados/pies
\pagestyle{fancy}
\rhead{UNED}
\lhead{Seminario II - Formación permanente del profesorado de secundaria/FP: El caso Erasmus+} 
\cfoot{\thepage}

\date{\today}

\begin{document}

% ---- Portada
\begin{center}
  \includegraphics[width=0.32\textwidth]{icono_uned.jpg}\par
  \vspace{0.3cm}
  {\bfseries\large Universidad Nacional de Educación a Distancia \par}
  {\scshape Facultad de Educación – Máster en Formación del Profesorado\par}
  \vspace{0.4cm}
  {\scshape Prácticum – Seminario II\par}
  \vspace{0.4cm}
  {\itshape\large Formación permanente del profesorado de secundaria/FP: El caso Erasmus+\par}
  \vspace{0.5cm}
  {\large Dustin Randel De Lara García \par}
  {\large \today \par}
\end{center}

\vspace{0.5cm}
\hrule

\renewcommand{\contentsname}{Índice de contenidos}
\tableofcontents

\section{Objetivos}\label{sec:objetivos}
El presente seminario tiene como objetivo principal presentar las distintas vías de formación permanente disponibles para el profesorado de educación secundaria y formación profesional. A lo largo del seminario se abordará el papel que desempeña la formación continua en el desarrollo profesional docente y las principales iniciativas que la impulsan.

Se prestará especial atención al programa Erasmus+, como ejemplo destacado de formación internacional orientada a la movilidad del profesorado. En particular, se analizarán las actividades de observación en el aula (job shadowing) y su relevancia para el intercambio de buenas prácticas entre centros europeos.

En los siguientes apartados se presentarán las características del proyecto Erasmus+ elegido como caso de estudio y la posible aplicación de las competencias adquiridas a la práctica docente, mediante la exposición de diversos supuestos teóricos.
\section{Características del proyecto}\label{sec:caracteristicas-del-proyecto}

\section{¿Porqué es interesante este proyecto?}\label{sec:porque-es-interesante}
\subsection{Aportación a la formación permanente del profesorado de secundaria}\label{subsec:aportacion-a-la-formacion-permanente-del-profesorado-de-secundaria}
\section{Transferencia del aprendizaje adquirido en Erasmus+ a la realidad de la práctica docente cotidiana}\label{sec:Transferencia-aprendizaje}
\section{¿Cómo crees que aplicarías las habilidades y conocimientos adquiridos en Erasmus+ en tu día a día?}\label{sec:aplicacion-dia-a-dia}
\section{Posibles contextos de aplicación de las habilidades adquiridas en Erasmus+}\label{sec:posibles-contextos-de-aplicación}

\newpage
\printbibliography
\end{document}
