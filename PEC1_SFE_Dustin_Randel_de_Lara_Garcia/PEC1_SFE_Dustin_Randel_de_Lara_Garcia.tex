\documentclass[11pt,a4paper]{article}

% ---- Márgenes
\usepackage[a4paper, left=2cm, right=2cm, top=2.5cm, bottom=2.5cm]{geometry}

% ---- Idioma y fuentes
\usepackage[utf8]{inputenc}
\usepackage[T1]{fontenc}
\usepackage[spanish, es-tabla]{babel}
\usepackage{lmodern}

% ---- Matemáticas y tablas
\usepackage{amsmath,amsfonts,amssymb}
\usepackage{tabularx,booktabs}

% ---- Gráficos
\usepackage{graphicx}
\usepackage{subcaption}
\usepackage{float}
\usepackage[table,xcdraw]{xcolor}
\definecolor{verdePastel}{RGB}{204, 235, 197}
\definecolor{verdeFresco}{RGB}{60,160,60}

% ---- Estilo y utilidades
\usepackage{fancyhdr}
\setlength{\headheight}{14pt}
\usepackage{csquotes}
\usepackage{parskip}   % separa párrafos sin sangría
\usepackage{array,longtable}
\renewcommand{\arraystretch}{1.4}
\newcolumntype{L}[1]{>{\raggedright\arraybackslash}p{#1}}
\newcolumntype{C}[1]{>{\centering\arraybackslash}p{#1}}
\newcolumntype{B}[1]{>{\raggedright\arraybackslash}p{#1}}

% ---- Bibliografía (mejor con biber)
\usepackage[
  backend=biber,        % o 'bibtex' si no puedes usar biber
  style=numeric,        % números [1], [2], ...
  sorting=none,         % <- orden según aparición en el texto
  maxbibnames=99
]{biblatex}

\addbibresource{referencias.bib}

% ---- Hipervínculos (cargar al final)
\usepackage{hyperref}
\hypersetup{
  pdfauthor={Dustin Randel De Lara García},
  pdftitle={PEC1_SFE_Dustin_Randel_de_Lara_Garcia},
  pdfkeywords={Sociedad, Familia, Educación, investigación},
  colorlinks=true,
  linkcolor=verdeFresco,
  urlcolor=verdeFresco,
  citecolor=verdeFresco
}

% ---- “Tabla” en lugar de “Cuadro” (por si acaso)
\addto\captionsspanish{\renewcommand{\tablename}{Tabla}}

% ---- Encabezados/pies
\pagestyle{fancy}
\rhead{UNED}
\lhead{PEC 1 - Implicación parental en Educación Secundaria} 
\cfoot{\thepage}

\date{\today}

\begin{document}

% ---- Portada
\begin{center}
  {\includegraphics[width=0.5\textwidth]{icono_uned.jpg}\par}
  \vspace{1cm}
  {\bfseries\LARGE Universidad Nacional de Educación a Distancia \par}
  \vspace{1cm}
  {\scshape\Large Facultad de Educación\par}
  \vspace{0.5cm}
  {\scshape\Large Máster en Formación del Profesorado\par}
  \vspace{0.5cm}
  {\scshape\Large Sociedad, Familia y Educación \par}
  \vspace{1.5cm}
  {\scshape\Huge Implicación Parental en Educación Secundaria \par}
  \vspace{1.5cm}
  {\itshape\Large PEC 1 \par}
  \vfill
  {\Large Dustin Randel De Lara García \par}
  \vfill
  {\Large \today \par}
  \thispagestyle{empty}
\end{center}

% ---- Índice
\newpage
\tableofcontents
\newpage

\section*{DECLARACIÓN JURADA DE AUTORÍA DE TRABAJO ACADÉMICO}\label{sec:declaracion-jurada-de-autoria}
\begin{tabularx}{\textwidth}{|l|X|}
\hline
Nombre y Apellidos: & Dusitn Randel de Lara García \\
\hline
\end{tabularx}
\vspace{0.5cm}

Hace constar que es el/la autor/a del trabajo y DECLARA que se hace públicamente responsable de sus contenidos y elaboración, y que no ha incurrido en plagio, ni ha utilizado una inteligencia artificial generativa para su redacción.

Si se demostrara lo contrario, la persona abajo firmante aceptará las medidas disciplinarias o sancionadoras que correspondan.

\newpage
\section{Evidencia de haber cumplimentado el cuestionario \enquote{Implicación Parental}}\label{sec:evidendia-cuestionario}
En mi caso envié el cuestionario a 3 padres de alumnos de Educación Secundaria. A continuación, se muestran los códigos que les proporcioné:
\begin{itemize}
\item  Padre 1: DRLG1
\item  Padre 2: DRLG2
\item  Padre 3: DRLG3
\end{itemize}

\section{Análisis del informe general de datos}\label{sec:analisis-informe-general}
\subsubsection*{¿Qué patrones o tendencias se observan en los datos recogidos sobre la implicación parental?}
En los datos se observan tres tendencias claras. Primero, un fuerte desequilibrio de género: la mayoría de respuestas proceden de las madres, muy por encima de los padres y de otras figuras tutelares. Segundo, la implicación es elevada en el hogar, con frecuencia alta en supervisar deberes, controlar pantallas, organizar horarios, hablar sobre el día a día y el futuro académico. Tercero, la participación en el centro es mucho más limitada: se acude poco a actividades colectivas y la comunicación con el profesorado suele ser puntual, mayoritariamente no presencial y ligada a problemas concretos.
\subsubsection*{¿Cómo se relacionan los resultados obtenidos con las teorías y modelos sobre la participación familiar en la educación?}
El mayor peso de las madres confirma la persistencia de una división generizada del trabajo familiar, en línea con los análisis sobre conciliación. La alta implicación en casa encaja con el ámbito de “learning at home” del modelo de Epstein y con la parentalidad positiva, centrada en apoyo y acompañamiento cotidiano. En cambio, la escasa presencia en el centro y una comunicación sobre todo reactiva sitúan a muchas familias en una participación básicamente informativa, más que decisoria o evaluativa, próxima a metas de logro parental orientadas a evitar el fracaso.
\subsubsection*{¿Qué implicaciones tienen los datos para la práctica docente en educación secundaria?}
Los datos tienen varias implicaciones para la práctica docente en secundaria. Primero, obligan a tener en cuenta la sobrecarga de las madres y la persistencia de una división generizada de los cuidados, por lo que el profesorado debería diversificar horarios, canales y mensajes para implicar también a padres y otras figuras familiares. Segundo, muestran que la relación con las familias se mantiene en un nivel principalmente informativo y reactivo. Por tanto, conviene diseñar proyectos y espacios estables (reuniones, comisiones mixtas) que refuercen la participación decisoria y evaluativa que Includ-ed vincula al éxito escolar. Finalmente, orientan la acción docente hacia una comunicación más preventiva, continua y bidireccional.
\subsubsection*{¿De qué modo podrían los futuros docentes favorecer una mayor colaboración y comunicación con las familias desde su función profesional?}
Para favorecer una mayor colaboración, los futuros docentes deberían tratar a las familias como aliadas educativas, no solo como receptoras de información. Esto implica pasar de un contacto reactivo a una comunicación preventiva y continua: tutorías planificadas, mensajes breves y claros, uso razonable de plataformas digitales y devoluciones periódicas sobre el progreso. Siguiendo propuestas como las de Epstein o Includ-ed, pueden abrir espacios en los que las familias opinen y evalúen aspectos de la vida escolar, y no solo reciban información. Finalmente, es importante tener en cuenta la desigual distribución de los cuidados y ofrecer horarios y formatos flexibles que faciliten la participación de todos los miembros de la familia.
\subsubsection*{Reseña propia y comentarios a las aportaciones de 2 compañeros}
\begin{figure}[H]
  \centering
  \includegraphics[width=1\textwidth]{reseña_propia.png}
  \caption{Reseña propia en el foro UNED.}
\end{figure}

\begin{figure}[H]
  \centering
  \includegraphics[width=1\textwidth]{Reseña_Seila.png}
  \caption{Reseña de Seila en el foro UNED.}
\end{figure}

\begin{figure}[H]
  \centering
  \includegraphics[width=1\textwidth]{Reseña_Francisco.png}
  \caption{Reseña de Francisco en el foro UNED.}
\end{figure}

\begin{figure}[H]
  \centering
  \includegraphics[width=1\textwidth]{Aportacion_1.png}
  \caption{Aportación 1 a la reseña de Seila.}
\end{figure}

\begin{figure}[H]
  \centering
  \includegraphics[width=1\textwidth]{Aportacion_2.png}
  \caption{Aportación 1 a la reseña de Francisco.}
\end{figure}
\nocite{rodriguez_bravo_sociedad_2025}
\printbibliography
\end{document}
