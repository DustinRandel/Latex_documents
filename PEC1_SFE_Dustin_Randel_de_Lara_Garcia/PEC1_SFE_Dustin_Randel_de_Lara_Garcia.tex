\documentclass[11pt,a4paper]{article}

% ---- Márgenes
\usepackage[a4paper, left=2cm, right=2cm, top=2.5cm, bottom=2.5cm]{geometry}

% ---- Idioma y fuentes
\usepackage[utf8]{inputenc}
\usepackage[T1]{fontenc}
\usepackage[spanish, es-tabla]{babel}
\usepackage{lmodern}

% ---- Matemáticas y tablas
\usepackage{amsmath,amsfonts,amssymb}
\usepackage{tabularx,booktabs}

% ---- Gráficos
\usepackage{graphicx}
\usepackage{subcaption}
\usepackage{float}
\usepackage[table,xcdraw]{xcolor}
\definecolor{verdePastel}{RGB}{204, 235, 197}
\definecolor{verdeFresco}{RGB}{60,160,60}

% ---- Estilo y utilidades
\usepackage{fancyhdr}
\setlength{\headheight}{14pt}
\usepackage{csquotes}
\usepackage{parskip}   % separa párrafos sin sangría
\usepackage{array,longtable}
\renewcommand{\arraystretch}{1.4}
\newcolumntype{L}[1]{>{\raggedright\arraybackslash}p{#1}}
\newcolumntype{C}[1]{>{\centering\arraybackslash}p{#1}}
\newcolumntype{B}[1]{>{\raggedright\arraybackslash}p{#1}}

% ---- Bibliografía (mejor con biber)
\usepackage[
  backend=biber,        % o 'bibtex' si no puedes usar biber
  style=numeric,        % números [1], [2], ...
  sorting=none,         % <- orden según aparición en el texto
  maxbibnames=99
]{biblatex}

\addbibresource{referencias.bib}

% ---- Hipervínculos (cargar al final)
\usepackage{hyperref}
\hypersetup{
  pdfauthor={Dustin Randel De Lara García},
  pdftitle={PEC1_SFE_Dustin_Randel_de_Lara_Garcia},
  pdfkeywords={Sociedad, Familia, Educación, investigación},
  colorlinks=true,
  linkcolor=verdeFresco,
  urlcolor=verdeFresco,
  citecolor=verdeFresco
}

% ---- “Tabla” en lugar de “Cuadro” (por si acaso)
\addto\captionsspanish{\renewcommand{\tablename}{Tabla}}

% ---- Encabezados/pies
\pagestyle{fancy}
\rhead{UNED}
\lhead{PEC 1 - Implicación parental en Educación Secundaria} 
\cfoot{\thepage}

\date{\today}

\begin{document}

% ---- Portada
\begin{center}
  {\includegraphics[width=0.5\textwidth]{icono_uned.jpg}\par}
  \vspace{1cm}
  {\bfseries\LARGE Universidad Nacional de Educación a Distancia \par}
  \vspace{1cm}
  {\scshape\Large Facultad de Educación\par}
  \vspace{0.5cm}
  {\scshape\Large Máster en Formación del Profesorado\par}
  \vspace{0.5cm}
  {\scshape\Large Sociedad, Familia y Educación \par}
  \vspace{1.5cm}
  {\scshape\Huge Implicación Parental en Educación Secundaria \par}
  \vspace{1.5cm}
  {\itshape\Large PEC 1 \par}
  \vfill
  {\Large Dustin Randel De Lara García \par}
  \vfill
  {\Large \today \par}
  \thispagestyle{empty}
\end{center}

% ---- Índice
\newpage
\tableofcontents
\newpage

\section*{DECLARACIÓN JURADA DE AUTORÍA DE TRABAJO ACADÉMICO}\label{sec:declaracion-jurada-de-autoria}
\begin{tabularx}{\textwidth}{|l|X|}
\hline
Nombre y Apellidos: & Dusitn Randel de Lara García \\
\hline
\end{tabularx}
\vspace{0.5cm}

Hace constar que es el/la autor/a del trabajo y DECLARA que se hace públicamente responsable de sus contenidos y elaboración, y que no ha incurrido en plagio, ni ha utilizado una inteligencia artificial generativa para su redacción.

Si se demostrara lo contrario, la persona abajo firmante aceptará las medidas disciplinarias o sancionadoras que correspondan.

\newpage
\section{Evidencia de haber cumplimentado el cuestionario \enquote{Implicación Parental}}\label{sec:evidendia-cuestionario}
En mi caso envié el cuestionario a 3 padres de alumnos de Educación Secundaria. A continuación, se muestran los códigos que les proporcioné:
\begin{itemize}
\item  Padre 1: DRLG1
\item  Padre 2: DRLG2
\item  Padre 3: DRLG3
\end{itemize}

\section{Análisis del informe general de datos}\label{sec:analisis-informe-general}
\subsubsection*{¿Qué patrones o tendencias se observan en los datos recogidos sobre la implicación parental?}

\subsubsection*{¿Cómo se relacionan los resultados obtenidos con las teorías y modelos sobre la participación familiar en la educación?}

\subsubsection*{¿Qué implicaciones tienen los datos para la práctica docente en educación secundaria?}

\subsubsection*{¿De qué modo podrían los futuros docentes favorecer una mayor colaboración y comunicación con las familias desde su función profesional?}

\newpage
\printbibliography
\end{document}
